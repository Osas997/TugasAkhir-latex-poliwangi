%==================================================================
% Ini adalah bab 5
% Silahkan edit sesuai kebutuhan, baik menambah atau mengurangi \section, \subsection
%==================================================================

\chapter[KESIMPULAN DAN SARAN]{\\ KESIMPULAN DAN SARAN}

\section{Kesimpulan}

Berdasarkan keseluruhan tahapan perancangan, implementasi, pengujian, dan pembahasan yang telah dilakukan dalam penelitian ini, dapat disimpulkan bahwa pengembangan chatbot analisis sentimen UMKM berbasis web dengan integrasi Retrieval-Augmented Generation (RAG) telah berhasil direalisasikan sesuai dengan tujuan yang telah ditetapkan. Sistem yang dikembangkan mampu berfungsi sebagai jembatan antara hasil analisis sentimen berbasis data kuantitatif dengan pemahaman pengguna melalui interaksi berbasis bahasa alami. Implementasi backend menggunakan NestJS terbukti memberikan struktur sistem yang modular, terorganisasi, dan mudah dipelihara, sementara integrasi LangChain.js memungkinkan mekanisme RAG berjalan secara efektif dengan memanfaatkan knowledge base berupa file JSON hasil analisis sentimen. Secara umum, penelitian ini menunjukkan bahwa pendekatan RAG dapat meningkatkan relevansi dan kontekstualitas jawaban chatbot dalam membantu pelaku UMKM memahami insight dari data sentimen media sosial.

Berdasarkan rumusan masalah yang telah ditetapkan pada Bab 1, maka kesimpulan spesifik dari penelitian ini adalah sebagai berikut:

\begin{itemize}
    \item Sistem API backend berbasis NestJS berhasil dirancang dan diimplementasikan dengan menyediakan layanan autentikasi berbasis JSON Web Token (JWT) serta endpoint khusus untuk chatbot analisis sentimen UMKM. Arsitektur modular NestJS memungkinkan pemisahan tanggung jawab antar komponen sistem, sehingga meningkatkan keterbacaan, skalabilitas, dan kemudahan pemeliharaan kode.

    \item Metode Retrieval-Augmented Generation (RAG) berhasil diintegrasikan menggunakan LangChain.js dengan memanfaatkan knowledge base berupa file JSON hasil analisis sentimen media sosial UMKM. Mekanisme ingestion, embedding, penyimpanan vektor, dan retrieval berjalan dengan baik sehingga chatbot mampu menghasilkan jawaban yang relevan, kontekstual, dan berbasis data terhadap pertanyaan pengguna.
\end{itemize}

\section{Saran}

Berdasarkan hasil penelitian yang telah dilakukan, terdapat beberapa saran yang dapat menjadi pertimbangan untuk pengembangan sistem lebih lanjut maupun penelitian lanjutan di masa mendatang. Saran-saran ini disusun berdasarkan keterbatasan penelitian yang telah diidentifikasi serta potensi pengembangan yang dapat meningkatkan kualitas dan cakupan sistem.


\begin{itemize}
    \item \textbf{Optimasi Parameter RAG Pipeline}

    Pengembangan lebih lanjut dapat dilakukan dengan mengimplementasikan \textit{dynamic k selection} pada proses \textit{similarity search}, sehingga jumlah dokumen yang diambil dapat menyesuaikan kompleksitas pertanyaan pengguna atau tingkat keyakinan (\textit{confidence}) hasil pencarian. Selain itu, parameter \textit{temperature} pada model LLM yang saat ini bernilai statis 0.7 dapat dibuat menjadi lebih fleksibel (\textit{configurable}) atau adaptif, di mana pertanyaan bersifat faktual menggunakan nilai temperature yang lebih rendah untuk menjaga konsistensi jawaban, sedangkan pertanyaan yang membutuhkan analisis atau insight kreatif dapat menggunakan temperature yang lebih tinggi. Dari sisi pemrosesan dokumen, strategi \textit{semantic chunking} dapat ditingkatkan dengan menerapkan teknik \textit{overlapping chunks} atau \textit{hierarchical chunking} guna meminimalkan risiko hilangnya informasi penting di batas antar potongan dokumen.

    \item \textbf{Evaluasi dan Validasi Lebih Lanjut}

    Pengujian sistem dapat diperluas melalui pelaksanaan \textit{User Acceptance Testing} (UAT) dengan melibatkan pelaku UMKM sebagai pengguna akhir, sehingga diperoleh umpan balik yang lebih representatif terkait aspek kegunaan (\textit{usability}) dan manfaat sistem dalam konteks nyata. Selain itu, kualitas jawaban chatbot dapat dievaluasi secara lebih objektif menggunakan metrik kuantitatif seperti BLEU, ROUGE, atau BERTScore untuk mengukur kesesuaian dan relevansi respons terhadap pertanyaan pengguna. Studi komparatif dengan pendekatan lain, seperti fine-tuning model atau penggunaan model LLM yang berbeda, juga dapat dilakukan untuk menganalisis trade-off antara kompleksitas implementasi, biaya komputasi, dan kualitas hasil yang diperoleh.

    \item \textbf{Pengembangan Conversational Memory dan Multi-Turn Interaction}

    Sistem saat ini masih memberikan jawaban dalam format teks yang bersifat satu arah tanpa kemampuan untuk mempertahankan konteks percakapan secara berkelanjutan. Oleh karena itu, pengembangan \textit{conversational memory} diperlukan agar chatbot mampu mengingat konteks interaksi sebelumnya dan memberikan respons yang lebih koheren terhadap pertanyaan lanjutan. Sebagai contoh, ketika pengguna menanyakan sentimen terhadap suatu brand tertentu dan kemudian bertanya ``Bagaimana dengan kompetitornya?'', chatbot diharapkan dapat memahami bahwa pertanyaan tersebut merujuk pada brand yang telah dibahas sebelumnya. Implementasi mekanisme percakapan multi-turn ini akan meningkatkan keluwesan interaksi, membuat alur dialog lebih natural, serta menjadikan chatbot lebih menyerupai asisten virtual yang cerdas dan kontekstual.
\end{itemize}
