%==================================================================
% Ini adalah bab 2
% Silahkan edit sesuai kebutuhan, baik menambah atau mengurangi \section, \subsection
%==================================================================

\chapter[TINJAUAN PUSTAKA]{\\ TINJAUAN PUSTAKA}

\section{Landasan Teori}

\subsection{Usaha Mikro, Kecil, dan Menengah (UMKM)}
Usaha Mikro, Kecil, dan Menengah (UMKM) merupakan salah satu pilar utama perekonomian Indonesia yang memiliki peran strategis dalam pertumbuhan ekonomi nasional, penyerapan tenaga kerja, dan pengurangan kemiskinan. UMKM diklasifikasikan berdasarkan skala usaha, nilai aset, dan omzet tahunan yang dimiliki. Sektor UMKM memberikan kontribusi signifikan terhadap Produk Domestik Bruto (PDB) Indonesia dan menyerap sebagian besar tenaga kerja nasional, menjadikannya tulang punggung perekonomian Indonesia.

UMKM memiliki karakteristik yang unik, antara lain: fleksibilitas dalam beradaptasi dengan perubahan pasar, kemampuan memanfaatkan sumber daya lokal, serta kedekatan dengan konsumen. Namun, UMKM juga menghadapi berbagai tantangan seperti keterbatasan modal, akses teknologi yang terbatas, kemampuan manajerial yang kurang, serta kesulitan dalam pemasaran dan branding. Di era digital saat ini, UMKM dituntut untuk dapat beradaptasi dengan teknologi informasi dan memanfaatkan platform digital untuk meningkatkan daya saing mereka.

Dalam konteks pemasaran digital, UMKM perlu memahami persepsi dan sentimen konsumen terhadap produk atau layanan mereka. Analisis sentimen dari media sosial dapat memberikan wawasan berharga tentang bagaimana konsumen merespons brand, produk, atau kampanye pemasaran yang dilakukan \citep{permana_sentimen_2023}. Namun, banyak pelaku UMKM yang belum memiliki kemampuan atau sumber daya untuk melakukan analisis data secara mendalam, sehingga diperlukan tools yang mudah digunakan dan dapat memberikan insight yang actionable.

\subsection{Media Sosial sebagai Platform Pemasaran}
Media sosial telah menjadi salah satu platform pemasaran yang paling efektif di era digital, terutama bagi UMKM yang memiliki keterbatasan budget untuk iklan konvensional. Platform seperti Instagram, Facebook, TikTok, dan Twitter menawarkan berbagai fitur yang memungkinkan pelaku usaha untuk berinteraksi langsung dengan konsumen, membangun brand awareness, serta melakukan promosi dengan biaya yang relatif terjangkau.

Instagram, sebagai salah satu platform media sosial yang paling populer di Indonesia, memiliki lebih dari 80\% penetrasi di kalangan pengguna internet Indonesia menurut laporan \citep{datareportal_digital_2025}. Platform ini menyediakan berbagai fitur seperti Instagram Feed, Stories, Reels, dan Shopping yang dapat dimanfaatkan oleh UMKM untuk mempromosikan produk mereka. Fitur komentar dan caption pada postingan Instagram menjadi sumber data yang kaya untuk memahami sentimen dan opini konsumen terhadap suatu brand atau produk.

\citep{utari_strategi_2022} menyatakan bahwa strategi pemasaran UMKM melalui Instagram di era pemasaran 4.0 harus memperhatikan beberapa aspek penting seperti konten visual yang menarik, storytelling yang autentik, interaksi aktif dengan followers, serta pemanfaatan fitur-fitur Instagram untuk meningkatkan engagement. Lebih lanjut, \citep{arviani_sosial_2021} menjelaskan bahwa social media marketing memberikan peluang sekaligus tantangan bagi UMKM lokal, terutama dalam masa pandemi COVID-19 di mana aktivitas bisnis banyak beralih ke platform digital.

Komentar, ulasan, dan mentions di media sosial merupakan bentuk Electronic Word of Mouth (eWOM) yang memiliki pengaruh signifikan terhadap keputusan pembelian konsumen. \citep{fitriani_tweeting_2023} menunjukkan bahwa sentimen di media sosial dapat mencerminkan kondisi ekonomi makro dan persepsi masyarakat terhadap berbagai isu. Oleh karena itu, analisis sentimen dari data media sosial menjadi penting untuk membantu UMKM memahami bagaimana konsumen mempersepsikan brand mereka dan mengidentifikasi area yang perlu diperbaiki.

\subsection{Analisis Sentimen}
Analisis sentimen, juga dikenal sebagai opinion mining, adalah proses komputasional untuk mengidentifikasi dan mengekstrak informasi subjektif dari teks, seperti pendapat, emosi, dan sikap penulis terhadap suatu topik, produk, atau layanan. Analisis sentimen umumnya mengklasifikasikan teks menjadi tiga kategori utama: positif, negatif, dan netral, meskipun beberapa pendekatan juga menggunakan klasifikasi yang lebih detail seperti sangat positif, positif, netral, negatif, dan sangat negatif.

\begin{enumerate}
  \item \textbf{Pendekatan Analisis Sentimen}

        Analisis sentimen dapat dilakukan melalui beberapa pendekatan utama yang masing-masing memiliki karakteristik, kelebihan, dan keterbatasan tersendiri. Pemilihan pendekatan yang tepat sangat bergantung pada kompleksitas data teks, ketersediaan data berlabel, serta tujuan analisis yang ingin dicapai. Secara umum, pendekatan analisis sentimen dapat diklasifikasikan menjadi tiga kategori utama, yaitu pendekatan berbasis leksikon, machine learning, dan deep learning.

        \begin{enumerate}
          \item \textbf{Lexicon-based approach} merupakan pendekatan yang menggunakan kamus sentimen yang berisi daftar kata-kata beserta polaritas sentimen yang telah ditentukan sebelumnya, seperti positif, negatif, atau netral. Pendekatan ini relatif sederhana dan tidak memerlukan data latih, sehingga mudah diimplementasikan. Namun, pendekatan berbasis leksikon memiliki keterbatasan dalam menangani teks yang kompleks, ambigu, atau mengandung sarkasme, serta kurang mampu memahami konteks kalimat secara menyeluruh.

          \item \textbf{Machine learning approach} memanfaatkan algoritma machine learning seperti Naive Bayes, Support Vector Machine (SVM), atau Random Forest yang dilatih menggunakan data berlabel untuk mengklasifikasikan sentimen. Pendekatan ini umumnya menghasilkan akurasi yang lebih baik dibandingkan pendekatan berbasis leksikon karena mampu mempelajari pola dari data. Sebagai contoh, \citep{permana_sentimen_2023} menggunakan Naive Bayes Classifier untuk menganalisis sentimen opini masyarakat terhadap UMKM di platform Twitter, yang menunjukkan efektivitas pendekatan ini dalam mengolah data teks media sosial.

          \item \textbf{Deep learning approach} menggunakan arsitektur neural network yang lebih kompleks, seperti Long Short-Term Memory (LSTM), Convolutional Neural Network (CNN), serta model berbasis transformer seperti BERT dan RoBERTa. Pendekatan ini memiliki kemampuan yang lebih baik dalam menangkap konteks, hubungan antar kata, serta makna semantik dalam teks yang panjang dan tidak terstruktur. Hal ini membuat pendekatan deep learning sangat cocok untuk analisis sentimen pada data media sosial yang bersifat dinamis dan kompleks. Sebagai contoh, \citep{ningrum_pengembangan_2025} memanfaatkan model BERT untuk melakukan analisis sentimen terhadap komentar TikTok pada UMKM, dengan hasil yang menunjukkan peningkatan performa dibandingkan metode tradisional.
        \end{enumerate}


  \item \textbf{Analisis Sentimen pada Media Sosial}

        Analisis sentimen pada media sosial memiliki tantangan tersendiri yang berbeda dibandingkan dengan teks formal, karena karakteristik bahasa yang digunakan cenderung bersifat informal dan tidak terstruktur. Teks pada media sosial sering kali mengandung penggunaan slang, singkatan, emoji, hashtag, serta variasi gaya bahasa yang tidak selalu mengikuti kaidah tata bahasa baku. Kondisi ini membuat proses pemrosesan dan interpretasi sentimen menjadi lebih kompleks, sehingga memerlukan pendekatan analisis yang tepat dan robust.

        Beberapa penelitian terdahulu menunjukkan bahwa meskipun memiliki tantangan, analisis sentimen pada media sosial tetap dapat memberikan hasil yang akurat dan bernilai. \citep{zaenab_kurnia_analisis_2024}, dalam penelitiannya mengenai analisis sentimen komentar kerja sama TikTok Shop dan Tokopedia di platform Instagram, menerapkan metode Naive Bayes Classifier dan berhasil memperoleh tingkat akurasi sebesar 84\%. Hasil tersebut menunjukkan bahwa algoritma machine learning masih mampu bekerja dengan baik dalam mengklasifikasikan sentimen pada data media sosial yang bersifat dinamis.

        Penelitian lain yang dilakukan oleh \citep{ibrahim_sentiment_2022} menganalisis sentimen pengguna terhadap akun Twitter dan Instagram Perpustakaan Nasional Indonesia. Hasil penelitian tersebut menunjukkan bahwa mayoritas sentimen yang muncul bersifat positif, yang mengindikasikan tingkat kepuasan pengguna terhadap layanan yang diberikan. Temuan ini menegaskan bahwa analisis sentimen pada media sosial tidak hanya berguna untuk memahami opini publik, tetapi juga dapat dimanfaatkan sebagai sumber umpan balik yang berharga dalam proses evaluasi dan peningkatan kualitas layanan.

\end{enumerate}

\subsection{Retrieval-Augmented Generation (RAG)}

Retrieval-Augmented Generation (RAG) merupakan sebuah pendekatan dalam Natural Language Processing (NLP) yang mengintegrasikan dua komponen utama, yaitu sistem retrieval untuk pengambilan informasi dan model generatif untuk pembangkitan teks. Pendekatan ini pertama kali diperkenalkan oleh \citep{lewis_retrieval-augmented_2021} sebagai solusi untuk meningkatkan akurasi, relevansi, dan keandalan jawaban pada tugas-tugas NLP yang bersifat knowledge-intensive, di mana model membutuhkan akses terhadap pengetahuan eksternal yang luas dan dinamis. Dengan mengombinasikan retrieval dan generation, RAG memungkinkan model menghasilkan jawaban yang tidak hanya koheren secara linguistik, tetapi juga didasarkan pada informasi faktual dari sumber eksternal.

\begin{enumerate}
  \item \textbf{Arsitektur RAG}

        \begin{figure}[h]
          \centering
          \includegraphics[width=0.8\textwidth]{rag-arsitektur.jpg}
          \caption{Arsitektur Retrieval-Augmented Generation (RAG)}
          \label{fig:rag-architecture}
        \end{figure}

        Arsitektur RAG terdiri dari dua pipeline utama yang bekerja secara terintegrasi, yaitu \textit{RAG Ingestion Pipeline} dan \textit{Prompt Elements Pipeline}, sebagaimana ditunjukkan pada Gambar \ref{fig:rag-architecture}. Kedua pipeline ini memiliki peran yang berbeda namun saling melengkapi dalam keseluruhan proses retrieval dan generation.

        RAG Ingestion Pipeline merupakan tahap awal yang bertujuan untuk menyiapkan dan membangun knowledge base yang akan digunakan dalam proses retrieval.Tahap pertama dalam pipeline ini adalah proses \textit{clean}. Pada tahap ini, data yang berasal dari berbagai sumber seperti dokumen teks, file JSON, maupun sumber data lainnya dibersihkan dari berbagai bentuk noise, termasuk karakter tidak relevan, format yang tidak konsisten, serta informasi yang tidak diperlukan. Proses pembersihan data ini sangat penting untuk memastikan kualitas data yang akan diproses pada tahap selanjutnya.

        Setelah data dibersihkan, tahap berikutnya adalah \textit{chunk}. Pada tahap ini, dokumen yang telah melalui proses pembersihan dipecah menjadi potongan-potongan kecil atau \textit{chunks}. Proses chunking dilakukan karena embedding model memiliki batasan panjang input, sehingga dokumen yang terlalu panjang tidak dapat diproses secara langsung. Selain itu, chunk yang lebih kecil memungkinkan proses retrieval yang lebih presisi, karena sistem dapat mengambil bagian dokumen yang paling relevan dengan query pengguna.

        Tahap selanjutnya adalah \textit{embed}, yaitu proses mengubah setiap chunk dokumen menjadi embedding vector menggunakan embedding model. Embedding merupakan representasi numerik berdimensi tinggi yang menangkap makna semantik dari teks. Dengan menggunakan embedding, sistem dapat merepresentasikan teks dalam ruang vektor sehingga memungkinkan perhitungan kesamaan semantik antar teks secara matematis.

        Embedding vector yang dihasilkan kemudian disimpan pada tahap \textit{vector database}. Pada tahap ini, setiap chunk beserta metadata-nya, seperti sumber dokumen, waktu pembuatan, atau kategori, disimpan dalam sebuah vector database. Vector database seperti Pinecone, Chroma, atau FAISS dirancang khusus untuk melakukan pencarian berbasis similarity secara efisien, bahkan pada skala data yang sangat besar. Database ini memungkinkan sistem untuk menemukan potongan dokumen yang paling relevan terhadap sebuah query dalam waktu yang singkat.

        Setelah knowledge base siap, proses berlanjut ke \textit{Prompt Elements Pipeline}, yaitu tahap yang berfokus pada pemrosesan query pengguna dan pembangkitan jawaban. Proses dimulai dari \textit{user input}, di mana sistem menerima pertanyaan atau permintaan informasi dari pengguna dalam bentuk teks natural.

        Selanjutnya, pada tahap \textit{user input embedding}, query pengguna diubah menjadi embedding vector menggunakan embedding model yang sama dengan yang digunakan pada RAG Ingestion Pipeline. Konsistensi penggunaan embedding model ini sangat penting untuk memastikan bahwa perhitungan kesamaan antara query dan dokumen dalam vector database dapat dilakukan secara akurat.

        Embedding query kemudian digunakan pada tahap \textit{vector database retrieval}, di mana sistem melakukan pencarian terhadap vector database untuk mengambil sejumlah \textit{top-K} dokumen atau chunk yang paling relevan berdasarkan metrik kesamaan, seperti cosine similarity atau distance metric lainnya. Chunk yang diambil ini dilengkapi dengan metadata yang memberikan konteks tambahan mengenai sumber informasi.

        Tahap berikutnya adalah \textit{prompt construction}, yaitu proses penggabungan antara query pengguna, chunk dokumen hasil retrieval, dan prompt template. Prompt template berfungsi untuk mengatur struktur prompt serta memberikan instruksi yang jelas kepada Large Language Model (LLM) mengenai bagaimana konteks yang diberikan harus digunakan dalam proses pembangkitan jawaban.

        Pada tahap \textit{LLM generation}, prompt yang telah dikonstruksi dikirimkan ke LLM untuk menghasilkan jawaban. LLM memanfaatkan kemampuan generatifnya untuk menyusun jawaban yang koheren, sekaligus menggunakan informasi kontekstual dari retrieved documents sebagai dasar pengetahuan.

        Tahap akhir dari pipeline ini adalah \textit{generated answer}, yaitu keluaran berupa jawaban dalam bahasa natural yang relevan dengan pertanyaan pengguna dan memiliki dasar faktual yang kuat karena ditopang oleh informasi dari knowledge base.

        Arsitektur RAG memiliki keunggulan dibandingkan dengan model generatif murni (\textit{pure generative model}) karena mampu menggabungkan kekuatan retrieval dan generation secara efektif. Proses retrieval memastikan bahwa jawaban yang dihasilkan bersumber dari informasi faktual yang dapat diperbarui, sementara model generatif memberikan fleksibilitas dalam menyajikan jawaban secara natural, koheren, dan sesuai dengan konteks pertanyaan pengguna.

  \item \textbf{Keunggulan RAG}

        Retrieval-Augmented Generation (RAG) memiliki sejumlah keunggulan yang menjadikannya lebih unggul dibandingkan dengan model generatif murni (\textit{pure generative model}), khususnya dalam konteks sistem tanya jawab yang membutuhkan akurasi tinggi dan ketergantungan pada pengetahuan faktual. Dengan mengintegrasikan mekanisme retrieval dan generation, RAG mampu mengatasi keterbatasan utama model generatif konvensional yang cenderung mengandalkan pengetahuan statis hasil pelatihan.

        \begin{enumerate}
          \item \textbf{Akurasi dan relevansi jawaban yang lebih tinggi}:
                RAG mampu menghasilkan jawaban yang lebih akurat dan relevan karena memanfaatkan knowledge base eksternal sebagai sumber informasi. Dengan adanya proses retrieval, model tidak hanya bergantung pada pengetahuan yang tersimpan dalam parameter hasil training, tetapi juga menggunakan konteks faktual yang diambil secara dinamis dari dokumen yang relevan. \citep{lewis_retrieval-augmented_2021} menunjukkan bahwa pendekatan ini secara signifikan meningkatkan performa pada tugas-tugas \textit{knowledge-intensive question answering} dibandingkan model generatif murni.

          \item \textbf{Kemampuan pembaruan pengetahuan tanpa pelatihan ulang}:
                Salah satu keunggulan utama RAG adalah kemampuannya untuk memperbarui pengetahuan sistem tanpa memerlukan proses pelatihan ulang model bahasa. Knowledge base dapat diperbarui secara independen sesuai dengan perubahan data atau informasi terbaru, sehingga sistem tetap adaptif terhadap dinamika pengetahuan. \citep{murtiyoso_systematic_2025} dalam studi tinjauan sistematisnya menegaskan bahwa fleksibilitas ini menjadikan RAG sangat cocok untuk domain yang informasinya sering berubah.

          \item \textbf{Pengurangan halusinasi}:
                Dengan menyediakan konteks eksplisit dari hasil retrieval, RAG membantu mengurangi risiko \textit{halusinasi}, yaitu kondisi di mana model menghasilkan informasi yang terdengar meyakinkan namun tidak berdasar fakta. \citep{pokhrel_practical_2025} menunjukkan bahwa integrasi retrieval dengan generation memungkinkan model menghasilkan jawaban yang lebih ter-grounded pada sumber data yang nyata, sehingga meningkatkan keandalan sistem chatbot berbasis website.

          \item \textbf{Transparansi dan verifikasi sumber informasi}:
                RAG memungkinkan sistem untuk melacak dan menyertakan metadata dari dokumen yang digunakan dalam proses pembangkitan jawaban. Hal ini memungkinkan pengguna atau pengembang untuk memverifikasi sumber informasi yang menjadi dasar jawaban, sehingga meningkatkan transparansi dan tingkat kepercayaan terhadap sistem. Pendekatan ini banyak digunakan dalam implementasi chatbot berbasis dokumen, seperti yang ditunjukkan oleh \citep{pratama_retrieval-augmented_2023} serta \citep{husain_development_2025}.

          \item \textbf{Efisiensi komputasi dan skalabilitas sistem}:
                Dengan memisahkan penyimpanan pengetahuan dari parameter model generatif, RAG tidak memerlukan model berukuran sangat besar untuk menyimpan seluruh informasi di dalam bobotnya. Pendekatan ini lebih efisien dari sisi komputasi dan penyimpanan, serta memungkinkan sistem untuk diskalakan dengan menambah data pada knowledge base tanpa meningkatkan kompleksitas model. \citep{vidivelli_efficiency-driven_2024} menunjukkan bahwa arsitektur RAG yang diorkestrasi menggunakan framework seperti LangChain dapat meningkatkan efisiensi pengembangan dan performa sistem chatbot.
        \end{enumerate}

  \item \textbf{Implementasi RAG dalam Berbagai Domain}

        Berbagai penelitian menunjukkan bahwa Retrieval-Augmented Generation (RAG) telah diimplementasikan secara luas untuk meningkatkan kemampuan model bahasa dalam memahami dan memanfaatkan pengetahuan spesifik domain. \citep{murtiyoso_systematic_2025} melakukan sebuah \textit{systematic review} terkait penggunaan RAG untuk meningkatkan \textit{domain-specific knowledge} pada Large Language Models. Hasil kajian tersebut menunjukkan bahwa pendekatan RAG efektif diterapkan pada berbagai aplikasi NLP, seperti \textit{question answering}, \textit{dialog systems}, dan \textit{information extraction}, karena kemampuannya dalam mengintegrasikan pengetahuan eksternal secara dinamis ke dalam proses pembangkitan jawaban.

        Penerapan RAG dalam berbagai domain dapat dilihat melalui sejumlah studi terdahulu yang mengadopsi pendekatan ini sesuai dengan kebutuhan masing-masing bidang. Beberapa contoh implementasi RAG dalam berbagai domain dijabarkan sebagai berikut.

        \begin{enumerate}
          \item \textbf{Helpdesk dan customer service}:
                \citep{pratama_pengembangan_2024} mengembangkan sistem helpdesk berbasis chatbot dengan menerapkan metode RAG untuk menjawab pertanyaan pengguna. Dalam penelitian tersebut, dokumen helpdesk digunakan sebagai knowledge base yang memungkinkan chatbot memberikan respons yang relevan dan kontekstual, sehingga meningkatkan kualitas layanan kepada pengguna.

          \item \textbf{Layanan akademik}:
                Pada domain layanan akademik, \citep{husain_development_2025} mengembangkan chatbot berbasis RAG yang dirancang untuk menjawab pertanyaan mahasiswa terkait layanan akademik. Dengan memanfaatkan dokumen akademik sebagai knowledge base, sistem yang dikembangkan mampu memberikan jawaban yang lebih akurat dan sesuai dengan kebijakan institusi.

          \item \textbf{Domain hukum}:
                Implementasi RAG juga diterapkan pada domain hukum yang memiliki karakteristik informasi kompleks dan spesifik. \citep{pratama_pengembangan_2024} menerapkan pendekatan RAG untuk penyediaan informasi hukum pidana Indonesia dengan menggunakan model LLaMA. Hasil penelitian tersebut menunjukkan bahwa RAG efektif dalam mendukung pengelolaan \textit{domain-specific knowledge base}, khususnya dalam konteks bahasa Indonesia.

          \item \textbf{Website-based chatbot}:
                \citep{pokhrel_practical_2025} mengimplementasikan RAG pada chatbot berbasis website dengan mengombinasikan proses \textit{web scraping}, vektorisasi dokumen, dan \textit{semantic search}. Penelitian ini memberikan panduan praktis mengenai tahapan implementasi RAG secara end-to-end serta menunjukkan bahwa pendekatan ini mampu meningkatkan relevansi jawaban chatbot pada lingkungan berbasis web.

          \item \textbf{Document question answering}:
                Pada konteks \textit{document question answering}, \citep{muludi_retrieval-augmented_2024} menerapkan pendekatan RAG dengan memanfaatkan Large Language Model yang diperkaya dengan konteks dokumen. Hasil penelitian tersebut menunjukkan bahwa penggunaan RAG mampu meningkatkan akurasi jawaban karena model memperoleh konteks tambahan yang relevan dari dokumen sumber.
        \end{enumerate}

  \item \textbf{LangChain sebagai Framework RAG}

        LangChain merupakan sebuah framework \textit{open-source} yang dirancang untuk mempermudah pengembangan aplikasi berbasis Large Language Models (LLM), khususnya aplikasi yang mengadopsi pendekatan Retrieval-Augmented Generation (RAG). Framework ini menyediakan berbagai abstraksi dan komponen modular yang membantu pengembang dalam membangun pipeline RAG secara terstruktur dan efisien. Beberapa komponen utama yang disediakan oleh LangChain meliputi \textit{document loaders} untuk memuat data dari berbagai sumber, \textit{text splitters} untuk melakukan chunking dokumen, \textit{vector stores} untuk menyimpan embedding, \textit{retrievers} untuk melakukan pencarian berbasis similarity, serta \textit{chains} yang berfungsi mengorkestrasi alur kerja kompleks yang melibatkan proses retrieval dan generation secara terintegrasi.

        Penggunaan LangChain dalam pengembangan sistem RAG juga telah dibuktikan efektif dalam berbagai penelitian. \citep{vidivelli_efficiency-driven_2024} menunjukkan bahwa pemanfaatan LangChain pada pengembangan chatbot kustom berbasis LLM mampu meningkatkan efisiensi proses pengembangan serta mempermudah integrasi dengan berbagai model LLM dan vector database. Selain itu, LangChain menyediakan dukungan lintas bahasa pemrograman, termasuk versi JavaScript yang dikenal sebagai LangChain.js. Keberadaan LangChain.js memungkinkan implementasi RAG dalam ekosistem Node.js, sehingga memberikan fleksibilitas yang lebih besar bagi pengembang dalam membangun aplikasi chatbot berbasis web dan sistem backend modern.
\end{enumerate}

\subsection{Chatbot dan Interaksi Bahasa Natural}

Chatbot merupakan program komputer yang dirancang untuk mensimulasikan percakapan dengan manusia melalui antarmuka berbasis teks maupun suara. Tujuan utama dari chatbot adalah memungkinkan terjadinya interaksi yang alami antara manusia dan sistem komputer dengan menggunakan bahasa natural sebagai medium komunikasi. Dalam implementasinya, chatbot memanfaatkan teknik Natural Language Processing (NLP) untuk memahami maksud pengguna, memproses konteks percakapan, serta menghasilkan respons yang sesuai dan relevan.

Seiring dengan perkembangan teknologi kecerdasan buatan, chatbot telah mengalami evolusi yang signifikan. Pada tahap awal, chatbot umumnya dibangun menggunakan pendekatan berbasis aturan yang mengandalkan skenario percakapan statis dan logika \textit{if-then}, sehingga memiliki keterbatasan dalam menangani variasi bahasa dan pertanyaan yang tidak terduga. Perkembangan selanjutnya menghadirkan chatbot yang mampu memanfaatkan teknik pembelajaran mesin dan model bahasa untuk memahami kemiripan semantik serta konteks percakapan secara lebih fleksibel \citep{kavaz_chatbot-based_2023}.

Dalam konteks modern, chatbot berbasis kecerdasan buatan semakin mengandalkan model bahasa berskala besar yang mampu memahami hubungan antar kata, makna semantik, serta konteks percakapan secara lebih mendalam. Integrasi pendekatan retrieval dan generation, seperti pada metode Retrieval-Augmented Generation (RAG), memungkinkan chatbot untuk tidak hanya menghasilkan respons yang natural secara linguistik, tetapi juga didukung oleh informasi faktual dari knowledge base eksternal. Pendekatan ini terbukti mampu meningkatkan akurasi, relevansi, dan keandalan jawaban dibandingkan model generatif murni \citep{lewis_retrieval-augmented_2021,murtiyoso_systematic_2025}.

Kemampuan chatbot dalam mendukung interaksi bahasa natural menjadikannya solusi yang banyak digunakan dalam berbagai sektor, seperti layanan pelanggan, pendidikan, layanan publik, dan sistem informasi berbasis web. Dalam konteks visualisasi dan interpretasi data, chatbot berbasis bahasa natural dapat berfungsi sebagai antarmuka yang efektif untuk membantu pengguna memahami data yang kompleks, terutama bagi pengguna yang tidak memiliki latar belakang statistik \citep{kavaz_chatbot-based_2023}. Penelitian lain menunjukkan bahwa perbedaan tingkat pemahaman pengguna terhadap grafik sering menimbulkan kesenjangan interpretasi (\textit{gulf of interpretation}), sehingga chatbot dapat berperan sebagai jembatan untuk menyederhanakan dan menjelaskan insight data secara kontekstual \citep{quadri_you_2024,knoll_gulf_2025}.

Dalam konteks bisnis dan UMKM, chatbot berbasis AI generatif juga berpotensi berfungsi sebagai \textit{virtual data analyst} yang membantu pengguna memahami informasi dan insight dari data tanpa memerlukan keahlian teknis khusus. Beberapa penelitian menunjukkan bahwa chatbot berbasis LLM dapat meningkatkan efektivitas sistem business intelligence, pengalaman pengguna, serta engagement pada platform digital \citep{salim_llm_2025,cempaka_influence_2025}.

\subsection{REST API}

REST (Representational State Transfer) API merupakan arsitektur komunikasi berbasis protokol HTTP yang digunakan untuk membangun layanan web (\textit{web services}) yang dapat diakses oleh berbagai jenis aplikasi klien. REST API dirancang berdasarkan seperangkat prinsip arsitektural yang memungkinkan sistem terdistribusi berkomunikasi secara efisien, ringan, dan fleksibel melalui protokol HTTP standar. Pendekatan ini banyak digunakan dalam pengembangan aplikasi modern karena kemudahannya dalam integrasi lintas platform serta dukungannya terhadap skalabilitas sistem.

Salah satu prinsip utama dalam REST API adalah pemisahan yang jelas antara sisi klien dan server (\textit{client-server architecture}), di mana klien bertanggung jawab terhadap antarmuka pengguna, sedangkan server menangani pemrosesan data dan logika bisnis. Selain itu, REST API bersifat \textit{stateless}, yang berarti setiap permintaan dari klien harus membawa seluruh informasi yang dibutuhkan untuk diproses oleh server tanpa bergantung pada penyimpanan status sesi sebelumnya. Karakteristik ini membuat REST API lebih mudah diskalakan dan dikelola. REST API juga mendukung mekanisme \textit{cacheable}, di mana respons dari server dapat ditandai untuk disimpan sementara oleh klien atau perantara jaringan guna meningkatkan performa dan mengurangi beban server. Prinsip lain yang penting adalah penggunaan \textit{uniform interface}, yaitu penerapan struktur URL yang konsisten serta penggunaan metode HTTP standar, sehingga API lebih mudah dipahami dan digunakan oleh pengembang. Selain itu, arsitektur REST memungkinkan penerapan \textit{layered system}, di mana berbagai lapisan seperti load balancer, cache, atau API gateway dapat diterapkan tanpa diketahui oleh klien, sehingga meningkatkan keamanan dan skalabilitas sistem.

Dalam implementasinya, REST API memanfaatkan metode HTTP untuk mendefinisikan operasi yang dapat dilakukan terhadap suatu resource. Metode \textit{GET} digunakan untuk mengambil data tanpa mengubah state sistem, sedangkan \textit{POST} digunakan untuk membuat resource baru di server. Metode \textit{PUT} berfungsi untuk memperbarui resource yang sudah ada atau membuat resource baru jika belum tersedia, sementara \textit{DELETE} digunakan untuk menghapus resource. Selain itu, metode \textit{PATCH} memungkinkan pembaruan sebagian (\textit{partial update}) terhadap resource tertentu. Penggunaan metode-metode HTTP ini memberikan semantik yang jelas terhadap setiap operasi yang dilakukan oleh klien, sehingga meningkatkan keterbacaan dan konsistensi API.

Dalam penelitian ini, REST API diimplementasikan menggunakan NestJS sebagai backend framework untuk membangun layanan yang modular dan mudah dipelihara. Penerapan REST API memungkinkan sistem backend berfungsi sebagai penghubung antara antarmuka pengguna, layanan pemrosesan data, serta integrasi dengan layanan eksternal. \citep{muhammad_development_2024} menunjukkan bahwa arsitektur REST API yang dibangun menggunakan NestJS mampu memberikan struktur sistem yang rapi, terorganisasi, dan mudah dikembangkan pada sistem e-wallet. Pendekatan serupa diterapkan dalam penelitian ini untuk membangun berbagai endpoint, seperti autentikasi pengguna, manajemen data hasil scraping, API Gateway untuk layanan eksternal seperti analisis sentimen berbasis aspek (ABSA) dan sistem rekomendasi, serta endpoint chatbot berbasis Retrieval-Augmented Generation (RAG). Dengan arsitektur REST API, komunikasi antar komponen sistem dapat dilakukan secara terstandarisasi, efisien, dan mendukung pengembangan sistem yang skalabel.


\subsection{NestJS sebagai Backend Framework}

NestJS merupakan sebuah framework Node.js yang dirancang dengan arsitektur modular dan ditulis menggunakan TypeScript, sehingga mendukung pengembangan aplikasi backend yang terstruktur dan mudah dipelihara. Framework ini mengadopsi konsep \textit{dependency injection} yang memungkinkan pengelolaan komponen sistem secara terpisah dan terorganisasi dengan baik. Menurut \citep{muhammad_development_2024}, karakteristik tersebut menjadikan NestJS cocok digunakan untuk membangun sistem backend berskala menengah hingga besar yang membutuhkan fleksibilitas, skalabilitas, dan maintainability yang tinggi.

Selain mendukung modularisasi, NestJS memiliki ekosistem yang luas dan aktif, dengan dukungan berbagai library serta integrasi yang memudahkan pengembangan layanan backend modern. Struktur proyek yang rapi dan konsisten memudahkan pengembang dalam melakukan pengembangan lanjutan, pengujian, serta pemeliharaan sistem dalam jangka panjang. NestJS juga menyediakan dukungan bawaan untuk pengembangan Application Programming Interface (API), termasuk integrasi dengan API Gateway serta penerapan berbagai metode autentikasi, seperti JSON Web Token (JWT), yang penting dalam pengelolaan akses dan keamanan sistem.

Dengan kemampuan tersebut, NestJS menjadi fondasi backend yang tepat untuk mendukung pengembangan sistem chatbot analisis sentimen pada penelitian ini. Arsitektur modular dan dukungan integrasi yang dimiliki NestJS memungkinkan pengelolaan komponen seperti layanan pemrosesan sentimen, integrasi model AI, serta penyediaan API secara efisien dan terstruktur, sehingga mendukung kinerja sistem secara keseluruhan.

\subsection{Model Fountain}

Model Fountain merupakan metodologi pengembangan perangkat lunak yang bersifat iteratif dan inkremental, yang pertama kali diperkenalkan oleh \citep{henderson-sellers_object-oriented_1990} sebagai alternatif terhadap model waterfall yang bersifat sekuensial dan rigid. Berbeda dengan model waterfall yang mengharuskan setiap tahap diselesaikan secara berurutan, model Fountain memberikan fleksibilitas yang lebih tinggi dengan memungkinkan terjadinya iterasi dan tumpang tindih antar tahapan pengembangan. Nama \textit{Fountain} digunakan karena alur pengembangannya menyerupai aliran air mancur, yang tidak hanya mengalir secara linear ke bawah, tetapi juga dapat mengalir kembali ke atas untuk proses penyempurnaan serta menyebar ke samping untuk memungkinkan pengembangan paralel.

\begin{figure}[h]
  \centering
  \includegraphics[width=0.4\textwidth]{fountain-model.jpg}
  \caption{Model Fountain}
  \label{fig:fountain-model}
\end{figure}

Gambar \ref{fig:fountain-model} memperlihatkan representasi visual dari model Fountain, di mana setiap fase pengembangan tidak bersifat terpisah secara kaku, melainkan saling terhubung dan dapat berlangsung secara iteratif. Diagram tersebut menggambarkan bahwa hasil dari suatu tahap dapat kembali memengaruhi tahap sebelumnya, sehingga proses evaluasi dan perbaikan dapat dilakukan secara berulang sepanjang siklus pengembangan sistem. Pendekatan ini sangat sesuai untuk pengembangan perangkat lunak yang kompleks dan dinamis, di mana kebutuhan sistem dapat berubah seiring waktu.

\subsubsection{Tahapan dalam Model Fountain}

Meskipun bersifat fleksibel dan iteratif, model Fountain tetap memiliki tahapan-tahapan utama yang menjadi acuan dalam pengembangan sistem. Tahapan-tahapan ini tidak harus dijalankan secara linier, melainkan dapat saling tumpang tindih dan dilakukan secara paralel sesuai dengan kebutuhan proyek. Adapun tahapan utama dalam model Fountain adalah sebagai berikut:

\begin{enumerate}
  \item \textbf{Analysis}:
        Tahap analisis bertujuan untuk memahami permasalahan yang akan diselesaikan oleh sistem serta konteks operasionalnya. Pada tahap ini dilakukan analisis terhadap kebutuhan pengguna, proses bisnis, serta batasan sistem yang akan dikembangkan.

  \item \textbf{Requirements Specification}:
        Tahap ini berfokus pada identifikasi dan dokumentasi kebutuhan sistem, baik kebutuhan fungsional maupun non-fungsional. Dalam model Fountain, spesifikasi kebutuhan tidak bersifat final dan dapat direvisi kapan saja apabila ditemukan kebutuhan baru atau perubahan pada kebutuhan pengguna.

  \item \textbf{Design}:
        Tahap perancangan meliputi desain arsitektur sistem, perancangan basis data, antarmuka pengguna, serta komponen-komponen sistem lainnya. Model Fountain memungkinkan proses desain dilakukan secara paralel untuk berbagai komponen, sehingga mempercepat pengembangan sistem secara keseluruhan.

  \item \textbf{Coding}:
        Tahap coding merupakan proses implementasi sistem berdasarkan desain yang telah dibuat. Pengembangan dilakukan secara inkremental, di mana modul atau fitur dikembangkan secara bertahap dan dapat langsung diuji serta dievaluasi sebelum melanjutkan ke modul berikutnya.

  \item \textbf{Testing and Integration}:
        Pada tahap ini dilakukan pengujian terhadap setiap komponen sistem serta proses integrasi antar komponen untuk memastikan sistem berfungsi sesuai dengan spesifikasi. Dalam model Fountain, pengujian dilakukan secara berkelanjutan selama proses pengembangan, bukan hanya pada tahap akhir.

  \item \textbf{Operation}:
        Tahap operation mencakup penggunaan sistem dalam lingkungan operasional nyata. Pada tahap ini, sistem mulai digunakan oleh pengguna dan kinerjanya diamati untuk mengidentifikasi potensi permasalahan atau kebutuhan perbaikan.

  \item \textbf{Maintenance}:
        Tahap maintenance berfokus pada pemeliharaan sistem, perbaikan kesalahan, serta penyesuaian sistem terhadap perubahan kebutuhan pengguna atau lingkungan operasional. Feedback dari pengguna menjadi masukan penting dalam tahap ini.

  \item \textbf{Evolution}:
        Tahap evolution merupakan kelanjutan dari maintenance, di mana sistem dikembangkan lebih lanjut melalui penambahan fitur baru atau peningkatan kemampuan sistem agar tetap relevan dan adaptif terhadap kebutuhan jangka panjang.
\end{enumerate}

\subsection{Black Box Testing}
Black Box Testing merupakan metode pengujian perangkat lunak yang berfokus pada pengujian fungsionalitas sistem tanpa mempertimbangkan struktur internal, desain, maupun implementasi kode program. Pendekatan ini juga dikenal sebagai \textit{functional testing} atau \textit{specification-based testing} karena proses pengujian dilakukan berdasarkan spesifikasi kebutuhan dan fungsionalitas sistem yang telah ditetapkan. Menurut \citep{khan_comparative_2021}, tujuan utama Black Box Testing adalah memastikan bahwa sistem menghasilkan output yang sesuai dengan input yang diberikan serta memenuhi requirement yang telah didefinisikan sebelumnya.

Dalam Black Box Testing, sistem dipandang sebagai sebuah “kotak hitam” di mana penguji hanya berfokus pada hubungan antara input dan output, tanpa perlu memahami bagaimana proses internal sistem bekerja. Pendekatan ini memungkinkan pengujian dilakukan dari perspektif pengguna akhir (\textit{end-user}), sehingga sangat efektif untuk mengevaluasi apakah sistem telah berfungsi sesuai dengan kebutuhan pengguna dan spesifikasi yang diharapkan. Penguji cukup memberikan berbagai skenario input dan kemudian memverifikasi apakah output yang dihasilkan telah sesuai dengan perilaku sistem yang diinginkan.

Lebih lanjut, \citep{khan_comparative_2021} menjelaskan bahwa Black Box Testing memiliki karakteristik yang menjadikannya metode pengujian yang praktis dan fleksibel. Metode ini tidak memerlukan pengetahuan tentang struktur internal program atau kode sumber, sehingga dapat dilakukan oleh penguji yang tidak memiliki latar belakang pemrograman. Fokus utama pengujian terletak pada validasi requirement dan fungsionalitas sistem, termasuk pemeriksaan terhadap antarmuka pengguna, perilaku sistem, serta aspek kinerja dasar. Melalui pendekatan ini, berbagai kesalahan seperti fungsi yang hilang, fungsi yang tidak sesuai, maupun kesalahan pada antarmuka dapat terdeteksi secara efektif sebelum sistem digunakan secara penuh.


\section{Penelitian Terkait}
Berikut adalah tabel perbandingan penelitian terkait yang relevan dengan pengembangan chatbot analisis sentimen UMKM berbasis RAG:


\begin{longtable}{|c|p{2.8cm}|p{2.5cm}|p{4cm}|p{4.5cm}|}
  \caption{Tabel perbandingan penelitian terkait} \label{tab:perbandingan_penelitian}                                                                                                                                                               \\
  \hline
  \textbf{No} & \textbf{Peneliti}              & \textbf{Teknologi}                    & \textbf{Judul}                                                     & \textbf{Fitur}                                                                        \\ \hline
  \endfirsthead

  \hline
  \textbf{No} & \textbf{Peneliti}              & \textbf{Teknologi}                    & \textbf{Judul}                                                     & \textbf{Fitur}                                                                        \\ \hline
  \endhead

  \hline
  \endfoot

  1           & Permana et al. (2023)          & Naive Bayes Classifier, Twitter API   & Sentimen Analisis Opini Masyarakat Terhadap UMKM pada Twitter      & Klasifikasi sentimen (positif, netral, negatif) dengan akurasi tinggi untuk data UMKM \\ \hline
  2           & Zaenab Kurnia et al. (2024)    & Naive Bayes Classifier, Instagram API & Analisis Sentimen Komentar TikTok Shop \& Tokopedia di Instagram   & Analisis sentimen komentar Instagram dengan akurasi 84\%                              \\ \hline
  3           & Ningrum et al. (2025)          & BERT, TikTok API                      & Bot Komentar Otomatis dengan Analisis Sentimen BERT untuk UMKM     & Sentimen berbasis deep learning + auto comment untuk engagement                       \\ \hline
  4           & Lewis et al. (2021)            & RAG, BERT, DPR                        & Retrieval-Augmented Generation for Knowledge-Intensive NLP Tasks   & Paper dasar RAG: retrieval + generation untuk Q\&A                                    \\ \hline
  5           & Pratama \& Sisephaputra (2024) & RAG, LangChain, Vector DB             & Sistem Helpdesk Chatbot dengan Metode RAG                          & Retrieval dokumen + respons berbasis knowledge base                                   \\ \hline
  6           & Husain et al. (2025)           & RAG, LLM, Vector Store                & Academic Services Chatbot Based on RAG                             & Retrieval dokumen akademik + generation jawaban akurat                                \\ \hline
  7           & Pokhrel et al. (2025)          & RAG, Web Scraping, Semantic Search    & Practical Application of RAG for Website-Based Chatbots            & RAG untuk chatbot website: scraping + semantic search                                 \\ \hline
  8           & Vidivelli et al. (2024)        & LangChain, RAG, Custom LLM            & Efficiency-Driven Chatbot: LangChain + RAG + LLM Fusion            & Integrasi RAG efisien + optimasi performa                                             \\ \hline
  9           & Muhammad \& Paputungan (2024)  & NestJS, REST API, TypeScript          & Backend Server with REST API Architecture for E-Wallet System      & Modular backend NestJS + dependency injection + REST API                              \\ \hline
  10          & Kavaz et al. (2023)            & NLP, Chatbot, Data Visualization      & Natural Language Interface for Data Visualization (Scoping Review) & Review chatbot untuk interpretasi visualisasi data                                    \\ \hline
\end{longtable}


Berdasarkan tabel \ref{tab:perbandingan_penelitian}, dapat diamati bahwa penelitian terkait analisis sentimen UMKM dan implementasi RAG telah berkembang pesat dalam beberapa tahun terakhir. Penelitian 1--3 menunjukkan berbagai pendekatan analisis sentimen untuk UMKM di media sosial, mulai dari metode klasik Naive Bayes hingga deep learning berbasis BERT. Penelitian 4--8 mendemonstrasikan implementasi RAG dalam berbagai domain seperti helpdesk, layanan akademik, dan website, membuktikan efektivitas metode ini untuk meningkatkan akurasi dan relevansi respons chatbot. Penelitian 9 memberikan justifikasi penggunaan NestJS sebagai backend framework yang robust dan scalable. Penelitian 10 mendukung konsep chatbot sebagai solusi untuk mengatasi kesenjangan pemahaman visualisasi data.

\section{Analisis Gap Penelitian}

Berdasarkan tinjauan terhadap penelitian-penelitian terkait pada Tabel \ref{tab:perbandingan_penelitian}, dapat diidentifikasi beberapa gap atau celah penelitian yang menjadi peluang bagi penelitian ini untuk memberikan kontribusi yang unik dan bernilai. Analisis gap penelitian diuraikan sebagai berikut:

\subsection{Gap 1: Tidak Adanya Integrasi Analisis Sentimen dengan Chatbot Interaktif}
Penelitian-penelitian analisis sentimen UMKM seperti \citep{permana_sentimen_2023}, \citep{zaenab_kurnia_analisis_2024}, dan \citep{ningrum_pengembangan_2025} umumnya fokus pada pengembangan model klasifikasi dan penyajian hasil dalam bentuk visualisasi statis seperti grafik batang, diagram lingkaran, atau tabel confusion matrix. Tidak ada penelitian yang menyediakan mekanisme interaktif berbasis chatbot untuk membantu pengguna, khususnya pelaku UMKM non-teknis, dalam memahami dan menginterpretasikan hasil analisis sentimen tersebut.

Pelaku UMKM yang tidak memiliki latar belakang statistik atau data science sering kali kesulitan untuk mengekstrak actionable insights dari visualisasi data. Mereka mungkin dapat melihat bahwa 58\% sentimen adalah positif, tetapi tidak tahu apa implikasinya terhadap strategi pemasaran mereka, bagaimana membandingkan dengan periode sebelumnya, atau aspek mana yang perlu diperbaiki. Penelitian ini mengisi gap dengan mengembangkan chatbot berbasis RAG yang dapat memberikan interpretasi kontekstual terhadap hasil analisis sentimen melalui interaksi bahasa natural, memungkinkan pengguna untuk bertanya dan mendapatkan insight yang disesuaikan dengan kebutuhan mereka.

\subsection{Gap 2: Implementasi RAG Tidak Spesifik untuk Domain Analisis Sentimen UMKM}
Meskipun beberapa penelitian seperti \citep{pratama_pengembangan_2024}, \citep{husain_development_2025}, dan \citep{pratama_retrieval-augmented_2023} telah menerapkan RAG dalam berbagai domain seperti helpdesk, layanan akademik, dan hukum pidana, belum ada penelitian yang mengimplementasikan RAG spesifik untuk domain analisis sentimen UMKM di Indonesia. Penelitian-penelitian RAG yang ada umumnya menggunakan dokumen teks panjang (PDF, Word, atau web content) sebagai knowledge base, bukan data terstruktur hasil analisis.

Penelitian ini mengisi gap dengan mengadaptasi metode RAG menggunakan knowledge base berbentuk file JSON yang berisi hasil analisis sentimen dari media sosial (Instagram) UMKM. Data JSON ini memiliki struktur yang berbeda dari dokumen teks biasa, yaitu berisi informasi terstruktur seperti kategori sentimen, jumlah komentar per kategori, tren temporal, keywords yang sering muncul, dan contoh komentar. Pendekatan ini memerlukan strategi retrieval dan generation yang disesuaikan untuk dapat mengekstrak dan menyajikan informasi dari data terstruktur secara efektif.

\subsection{Gap 3: Kombinasi Teknologi Modern untuk Sistem Backend yang Terstruktur}
Penelitian \citep{muhammad_development_2024} menunjukkan keunggulan NestJS dalam pembangunan REST API yang terstruktur, modular, dan terukur untuk aplikasi e-wallet. NestJS menawarkan arsitektur yang solid dengan dependency injection, middleware support, dan integration dengan berbagai libraries. Namun, belum ada penelitian yang mengintegrasikan NestJS dengan framework AI generatif seperti LangChain.js untuk keperluan chatbot berbasis RAG.

LangChain.js adalah framework yang powerful untuk membangun aplikasi LLM, tetapi implementasinya dalam ekosistem Node.js dengan backend framework enterprise-grade seperti NestJS belum banyak dieksplorasi dalam literatur akademik Indonesia. Penelitian ini mengisi gap dengan memberikan blueprint arsitektur teknis yang menggabungkan NestJS sebagai backend framework dan LangChain.js untuk orkestrasi RAG pipeline. Integrasi ini mencakup pengelolaan service layers, dependency injection untuk LLM components, error handling, dan API design yang RESTful untuk endpoint chatbot.