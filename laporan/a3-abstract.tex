%==================================================================
% Ini adalah abstrak dalam bahasa inggris 
%==================================================================

%% DILARANG EDIT BAGIAN INI
\clearpage
\phantomsection
\addcontentsline{toc}{chapter}{ABSTRACT}

\begin{center}
	\textbf{\large{\judulen}}\\[0.5cm]
\end{center}

\vspace{1cm}

\hspace{-1.5cm}\begin{tabular}{ll}
	\textit{By}                      & : \penulis                        \\
	\textit{Student Identity Number} & : \nim                            \\
	\textit{Supervisor}              & : 1. \pembimbingsatu              \\
	                                 & \hspace{0.15cm} 2. \pembimbingdua \\
\end{tabular}

\vspace{1cm}

\begin{center}
	\textbf{ABSTRACT}\\[0.5cm]
\end{center}
%% DILARANG EDIT BAGIAN INI

%% edit bagian ini
\textit{
	Micro, Small, and Medium Enterprises (MSMEs) in Indonesia increasingly utilize social media as a marketing channel, generating valuable public opinion data that can be analyzed through sentiment analysis. However, the results of such analysis are commonly presented in the form of charts and tables that are difficult for MSME actors to interpret due to limited data literacy and understanding of information technology. This study aims to develop a web-based sentiment analysis chatbot capable of presenting analysis results in an interactive and easily understandable natural language format using a Retrieval-Augmented Generation (RAG) approach.The system was developed using the Fountain Model, which supports iterative and parallel processes through the stages of requirement analysis, design, implementation, and testing. The backend was built using NestJS with REST API services. RAG integration was implemented using LangChain.js by utilizing embedding models, a Large Language Model (LLM), and vector storage in PostgreSQL with the pgvector extension. System testing was conducted using Black Box Testing to verify functionality and to evaluate the relevance of chatbot responses to the sentiment analysis knowledge base. The results show that the chatbot is able to present sentiment analysis information in contextual, data-driven natural language responses that are easy for MSME actors to understand without relying on statistical visualizations. This approach helps bridge the gap between quantitative data and user comprehension in the context of MSME digitalization.
}\\[0.6cm]
%% edit sampai sini

%% DILARANG EDIT BAGIAN INI
\noindent \textit{Key words: sentiment analysis, chatbot, Retrieval-Augmented Generation, MSMEs, NestJS.}
%% DILARANG EDIT BAGIAN INI