%==================================================================
% Ini adalah abstrak dalam bahasa inggris 
%==================================================================

%% DILARANG EDIT BAGIAN INI
\clearpage
\phantomsection
\addcontentsline{toc}{chapter}{ABSTRACT}
\begin{center}
    \textbf{\large{\judulen}}\\[0.5cm]
    by:\\
    \penulis\\
    NIM: \nim\\[2em]
    \textbf{ABSTRACT}\\[0.5cm]
\end{center}
%% DILARANG EDIT BAGIAN INI

%% edit bagian ini
\textit{
    Small and Medium Enterprises (SMEs) in Indonesia are increasingly utilizing social media as a platform for marketing and customer interaction, generating valuable public opinion data that can provide strategic insights through sentiment analysis. However, the results of sentiment analysis are typically presented in the form of statistical visualizations such as charts and tables, which are not always easy for SME practitioners to interpret due to limited data literacy and understanding of digital information. Previous studies have indicated that limitations in IT-related education are among the key challenges in SMEs' digital readiness, making the interpretation of data-driven visual information a significant obstacle. This study aims to develop a web-based sentiment analysis chatbot for SMEs that presents analysis results interactively and in an easily understandable manner through a natural language approach by integrating Retrieval-Augmented Generation (RAG). The system was developed using the Fountain Model, which supports iterative and parallel processes, including requirements analysis, design, implementation, and system testing. The backend system was built using NestJS and REST APIs, while RAG integration was implemented using LangChain.js with embedding models, a Large Language Model (LLM), and vector storage in PostgreSQL with the \textit{pgvector} extension.  System evaluation was conducted through Black Box Testing to verify service functionality and to assess the relevance of chatbot responses based on the sentiment analysis knowledge base. The results demonstrate that the system is capable of delivering contextual, data-driven responses in natural language form, thereby improving SMEs' accessibility to and understanding of sentiment analysis results without relying on the interpretation of complex visualizations. This approach contributes to bridging the gap between quantitative data and user comprehension in the context of SME digital practices.
}\\[0.6cm]
%% edit sampai sini

%% DILARANG EDIT BAGIAN INI
\noindent \textit{Key words: sentiment analysis, chatbot, Retrieval-Augmented Generation, SMEs, NestJS.}
%% DILARANG EDIT BAGIAN INI