%==================================================================
% Ini adalah abstrak dalam bahasa indonesia 
%==================================================================

%% DILARANG EDIT BAGIAN INI
\clearpage
\phantomsection
\addcontentsline{toc}{chapter}{ABSTRAK}
\begin{center}
    \textbf{\large{\judulid}}\\[0.5cm]
    Oleh\\
    \penulis\\
    NIM: \nim\\[2em]
    \textbf{ABSTRAK}\\[0.5cm]
\end{center}
%% DILARANG EDIT BAGIAN INI

%% edit bagian ini
Usaha Mikro, Kecil, dan Menengah (UMKM) di Indonesia semakin memanfaatkan media sosial sebagai sarana pemasaran dan interaksi dengan pelanggan, yang menghasilkan data opini publik bernilai strategis untuk dianalisis melalui analisis sentimen. Namun, hasil analisis tersebut umumnya disajikan dalam bentuk visualisasi statistik seperti grafik dan tabel yang tidak selalu mudah dipahami oleh pelaku UMKM akibat keterbatasan literasi data dan pemahaman terhadap informasi digital. Kondisi ini sejalan dengan temuan penelitian sebelumnya yang menunjukkan bahwa aspek \textit{IT-education} merupakan salah satu kelemahan utama dalam kesiapan digital UMKM, sehingga interpretasi terhadap informasi berbasis visualisasi data menjadi tantangan tersendiri. Penelitian ini bertujuan mengembangkan chatbot analisis sentimen UMKM berbasis web yang mampu menyajikan hasil analisis secara interaktif dan mudah dipahami melalui pendekatan bahasa alami dengan integrasi \textit{Retrieval-Augmented Generation (RAG)}. Pengembangan sistem dilakukan menggunakan Model Fountain yang mendukung proses iteratif dan paralel, mencakup tahap analisis kebutuhan, desain, implementasi, serta pengujian sistem. Backend sistem dibangun menggunakan \textit{NestJS} dan REST API, sementara integrasi RAG diimplementasikan menggunakan \textit{LangChain.js} dengan pemanfaatan model \textit{embedding}, \textit{Large Language Model (LLM)}, serta penyimpanan vektor menggunakan \textit{PostgreSQL} dengan ekstensi \textit{pgvector}. Pengujian sistem dilakukan melalui \textit{Black Box Testing} terhadap fungsionalitas layanan serta evaluasi relevansi jawaban chatbot terhadap \textit{knowledge base} hasil analisis sentimen. Hasil penelitian menunjukkan bahwa sistem mampu menyajikan informasi analisis sentimen dalam bentuk respons bahasa alami yang kontekstual dan berbasis data aktual, sehingga meningkatkan aksesibilitas dan pemahaman pelaku UMKM terhadap hasil analisis tanpa bergantung pada interpretasi grafik atau laporan statistik. Dengan demikian, pendekatan ini berkontribusi dalam menjembatani kesenjangan antara data kuantitatif dan pemahaman pengguna pada konteks digital UMKM.\\ [0.6cm]

%% edit sampai sini

%% DILARANG EDIT BAGIAN INI
\noindent Kata kunci: analisis sentimen, chatbot, \textit{Retrieval-Augmented Generation} , UMKM, NestJS.
%% DILARANG EDIT BAGIAN INI