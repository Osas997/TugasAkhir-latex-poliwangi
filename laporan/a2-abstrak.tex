%==================================================================
% Ini adalah abstrak dalam bahasa indonesia 
%==================================================================

%% DILARANG EDIT BAGIAN INI
\clearpage
\phantomsection
\addcontentsline{toc}{chapter}{ABSTRAK}
\begin{center}
    \textbf{\large{\judulid}}\\[0.5cm]
    Oleh\\
    \penulis\\
    NIM: \nim\\[2em]
    \textbf{ABSTRAK}\\[0.5cm]
\end{center}
%% DILARANG EDIT BAGIAN INI

%% edit bagian ini
Usaha Mikro, Kecil, dan Menengah (UMKM) di Indonesia semakin memanfaatkan media sosial sebagai sarana pemasaran dan interaksi dengan pelanggan, sehingga menghasilkan data opini publik yang berpotensi memberikan insight strategis melalui analisis sentimen. Namun, hasil analisis sentimen umumnya disajikan dalam bentuk visualisasi statistik yang relatif sulit dipahami oleh sebagian pelaku UMKM karena keterbatasan literasi data. Penelitian ini bertujuan mengembangkan chatbot analisis sentimen UMKM berbasis web yang mampu menyajikan hasil analisis secara interaktif dan mudah dipahami melalui pendekatan bahasa alami dengan integrasi Retrieval-Augmented Generation (RAG). Metode penelitian yang digunakan adalah pengembangan sistem menggunakan Model Fountain yang mendukung proses iteratif dan paralel, meliputi analisis kebutuhan, spesifikasi, desain, implementasi, serta pengujian dan integrasi sistem. Backend dikembangkan menggunakan framework NestJS dengan arsitektur modular, REST API berbasis JSON Web Token (JWT), serta pengelolaan data hasil analisis sentimen media sosial Instagram dalam format JSON. Integrasi RAG diimplementasikan menggunakan LangChain.js dengan pemanfaatan embedding model dan Large Language Model (LLM), serta penyimpanan vektor pada PostgreSQL dengan ekstensi \textit{pgvector} untuk mendukung pencarian berbasis kemiripan semantik. Pengujian sistem dilakukan menggunakan metode \textit{Black Box Testing} terhadap fungsionalitas endpoint API dan pengujian relevansi jawaban chatbot. Hasil penelitian menunjukkan bahwa seluruh skenario pengujian fungsional REST API berhasil dijalankan dengan baik, serta chatbot mampu menghasilkan jawaban yang kontekstual, relevan, dan berbasis fakta dari \textit{knowledge base} analisis sentimen. Dengan demikian, sistem chatbot berbasis RAG ini terbukti dapat meningkatkan aksesibilitas dan pemanfaatan hasil analisis sentimen bagi pelaku UMKM, serta berpotensi dikembangkan lebih lanjut melalui penerapan parameter dinamis dan memori percakapan untuk interaksi multi-turn yang lebih adaptif.\\ [0.6cm]

%% edit sampai sini

%% DILARANG EDIT BAGIAN INI
\noindent Kata kunci: analisis sentimen, chatbot, Retrieval-Augmented Generation, UMKM, NestJS.
%% DILARANG EDIT BAGIAN INI