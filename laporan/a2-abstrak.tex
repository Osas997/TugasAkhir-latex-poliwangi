%==================================================================
% Ini adalah abstrak dalam bahasa indonesia 
%==================================================================

%% DILARANG EDIT BAGIAN INI
\clearpage
\phantomsection
\addcontentsline{toc}{chapter}{ABSTRAK}
\begin{center}
    \textbf{\large{\judulid}}\\[0.5cm]
    Oleh\\
    \penulis\\
    NIM: \nim\\[2em]
    \textbf{ABSTRAK}\\[0.5cm]
\end{center}
%% DILARANG EDIT BAGIAN INI

%% edit bagian ini
Usaha Mikro, Kecil, dan Menengah (UMKM) di Indonesia semakin memanfaatkan media sosial sebagai sarana pemasaran dan interaksi dengan pelanggan, yang pada akhirnya menghasilkan data opini publik bernilai strategis untuk dianalisis melalui analisis sentimen. Namun, hasil analisis tersebut umumnya disajikan dalam bentuk visualisasi statistik seperti grafik dan tabel yang relatif sulit dipahami oleh sebagian pelaku UMKM akibat keterbatasan literasi data. Penelitian ini bertujuan mengembangkan chatbot analisis sentimen UMKM berbasis web yang mampu menyajikan hasil analisis secara interaktif dan mudah dipahami melalui pendekatan bahasa alami dengan integrasi \textit{Retrieval-Augmented Generation (RAG)}. Pengembangan sistem dilakukan menggunakan Model Fountain yang mendukung proses iteratif dan paralel, mencakup tahap analisis kebutuhan, spesifikasi, desain, implementasi, serta pengujian dan integrasi sistem. Backend dikembangkan menggunakan framework NestJS dengan arsitektur modular, REST API berbasis \textit{JSON Web Token (JWT)}, serta pengelolaan data hasil analisis sentimen media sosial Instagram dalam format JSON. Integrasi RAG diimplementasikan menggunakan LangChain.js dengan pemanfaatan model embedding dan \textit{Large Language Model (LLM)}, serta penyimpanan vektor pada PostgreSQL dengan ekstensi \textit{pgvector} untuk mendukung pencarian berbasis kemiripan semantik. Pengujian sistem dilakukan melalui metode \textit{Black Box Testing} terhadap fungsionalitas endpoint API dan relevansi jawaban chatbot. Hasil penelitian menunjukkan bahwa seluruh skenario pengujian fungsional REST API berhasil dijalankan dengan baik, serta chatbot mampu menghasilkan jawaban yang kontekstual, relevan, dan berbasis fakta dari \textit{knowledge base} analisis sentimen, sehingga sistem ini terbukti meningkatkan aksesibilitas dan pemanfaatan hasil analisis sentimen bagi pelaku UMKM serta memiliki potensi pengembangan lebih lanjut, seperti penerapan parameter dinamis dan memori percakapan untuk interaksi multi-turn yang lebih adaptif. \\[0.6cm]
\\ [0.6cm]

%% edit sampai sini

%% DILARANG EDIT BAGIAN INI
\noindent Kata kunci: analisis sentimen, chatbot, \textit{Retrieval-Augmented Generation} , UMKM, NestJS.
%% DILARANG EDIT BAGIAN INI