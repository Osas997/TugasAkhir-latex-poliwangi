%==================================================================
% Ini adalah abstrak dalam bahasa indonesia 
%==================================================================

%% DILARANG EDIT BAGIAN INI
\clearpage
\phantomsection
\addcontentsline{toc}{chapter}{ABSTRAK}
\begin{center}
	\textbf{\large{Pengembangan Chatbot Analisis Sentimen UMKM Berbasis Web Dengan Integrasi  Model \textit{RAG} Untuk Pencarian  Dan  \textit{Answer Generation}}}\\[0.5cm]
\end{center}

\vspace{1cm}

\hspace{-1.5cm}\begin{tabular}{ll}
	Nama mahasiswa & : \penulis                        \\
	NIM            & : \nim                            \\
	Pembimbing     & : 1. \pembimbingsatu              \\
	               & \hspace{0.15cm} 2. \pembimbingdua \\
\end{tabular}

\vspace{1cm}

\begin{center}
	\textbf{ABSTRAK}\\[0.5cm]
\end{center}
%% DILARANG EDIT BAGIAN INI

%% edit bagian ini
Usaha Mikro, Kecil, dan Menengah (UMKM) di Indonesia semakin memanfaatkan media sosial sebagai sarana pemasaran yang menghasilkan data opini publik bernilai strategis untuk dianalisis melalui analisis sentimen. Namun, hasil analisis umumnya disajikan dalam bentuk grafik dan tabel yang sulit dipahami oleh pelaku UMKM akibat keterbatasan literasi data dan pemahaman teknologi informasi. Penelitian ini bertujuan mengembangkan chatbot analisis sentimen berbasis web yang mampu menyajikan hasil analisis dalam bentuk bahasa alami yang interaktif dan mudah dipahami menggunakan pendekatan \textit{Retrieval-Augmented Generation} (RAG). Pengembangan sistem menggunakan Model Fountain yang mendukung proses iteratif dan paralel melalui tahap analisis kebutuhan, perancangan, implementasi, dan pengujian. Backend sistem dibangun menggunakan \textit{NestJS} dengan layanan REST API. Integrasi RAG diimplementasikan menggunakan \textit{LangChain.js} dengan pemanfaatan model \textit{embedding}, \textit{Large Language Model} (LLM), serta penyimpanan vektor pada \textit{PostgreSQL} dengan ekstensi \textit{pgvector}. Pengujian dilakukan menggunakan \textit{Black Box Testing} untuk memastikan fungsionalitas sistem serta evaluasi relevansi jawaban chatbot terhadap \textit{knowledge base} hasil analisis sentimen. Hasil penelitian menunjukkan bahwa chatbot mampu menyajikan informasi analisis sentimen dalam bentuk respons bahasa alami yang kontekstual, berbasis data aktual, dan mudah dipahami oleh pelaku UMKM tanpa bergantung pada interpretasi visualisasi statistik. Pendekatan ini membantu menjembatani kesenjangan antara data kuantitatif dan pemahaman pengguna dalam konteks digital UMKM.\\[0.6cm]

%% edit sampai sini

%% DILARANG EDIT BAGIAN INI
\noindent Kata kunci: analisis sentimen, chatbot, \textit{Retrieval-Augmented Generation} , UMKM, NestJS.
%% DILARANG EDIT BAGIAN INI