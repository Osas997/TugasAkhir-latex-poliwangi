%==================================================================
% Ini adalah bab 5
% Silahkan edit sesuai kebutuhan, baik menambah atau mengurangi \section, \subsection
%==================================================================

\chapter[KESIMPULAN DAN SARAN]{\\ KESIMPULAN DAN SARAN}

\section{Kesimpulan}

Berdasarkan seluruh tahapan penelitian, pengembangan chatbot analisis sentimen UMKM berbasis web dengan integrasi Retrieval-Augmented Generation (RAG) berhasil diwujudkan sesuai tujuan. Berdasarkan rumusan masalah pada Bab 1, kesimpulan spesifik penelitian ini adalah sebagai berikut:

\begin{enumerate}
    \item Sistem API backend berbasis NestJS berhasil dirancang dan diimplementasikan dengan layanan autentikasi berbasis JSON Web Token (JWT) serta endpoint khusus untuk chatbot analisis sentimen UMKM. Arsitektur modular NestJS mendukung pemisahan tanggung jawab antar komponen, sehingga meningkatkan keterbacaan, skalabilitas, dan kemudahan pemeliharaan kode.

    \item Metode Retrieval-Augmented Generation (RAG) berhasil diintegrasikan melalui LangChain.js dengan memanfaatkan knowledge base berupa file JSON hasil analisis sentimen media sosial UMKM. Proses ingestion, embedding, penyimpanan vektor, dan retrieval berjalan baik, sehingga chatbot mampu menghasilkan jawaban yang relevan, kontekstual, dan berbasis data terhadap pertanyaan pengguna.
\end{enumerate}


\section{Saran}

Dengan melihat hasil penelitian yang telah dilakukan, ada beberapa saran yang dapat menjadi pertimbangan untuk pengembangan sistem lebih lanjut maupun penelitian lanjutan di masa mendatang.

\begin{itemize}
    \item \textbf{Optimasi Parameter RAG Pipeline}

          Implementasi \textit{dynamic k selection} pada \textit{similarity search} agar jumlah dokumen yang diambil menyesuaikan kompleksitas pertanyaan. Selain itu, parameter \textit{temperature} LLM dapat dibuat adaptif untuk menyeimbangkan konsistensi jawaban faktual dan kreativitas analisis. Strategi \textit{semantic chunking} juga dapat ditingkatkan melalui \textit{overlapping} atau \textit{hierarchical chunking} guna mengurangi kehilangan informasi di batas potongan dokumen.

    \item \textbf{Evaluasi dan Validasi Lebih Lanjut}

          Pengujian dapat diperluas melalui \textit{User Acceptance Testing} (UAT) dengan pelaku UMKM untuk memperoleh umpan balik yang lebih representatif. Kualitas jawaban chatbot juga dapat dievaluasi menggunakan metrik seperti BLEU, ROUGE, atau BERTScore, serta dibandingkan dengan pendekatan fine-tuning atau model LLM lain.

    \item \textbf{Pengembangan Conversational}

          Diperlukan pengembangan \textit{conversational memory} agar chatbot dapat mempertahankan konteks percakapan dan merespons pertanyaan lanjutan secara tepat.
\end{itemize}

