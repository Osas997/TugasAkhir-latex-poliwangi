%==================================================================
% Ini adalah abstrak dalam bahasa inggris 
%==================================================================

%% DILARANG EDIT BAGIAN INI
\clearpage
\phantomsection
\addcontentsline{toc}{chapter}{ABSTRACT}
\begin{center}
    \textbf{\large{\judulen}}\\[0.5cm]
    by:\\
    \penulis\\
    NIM: \nim\\[2em]
    \textbf{ABSTRACT}\\[0.5cm]
\end{center}
%% DILARANG EDIT BAGIAN INI

%% edit bagian ini
\textit{Small and Medium Enterprises (SMEs) in Indonesia are increasingly utilizing social media as a marketing and customer interaction platform, generating large volumes of public opinion data that can provide strategic insights through sentiment analysis. However, sentiment analysis results are generally presented in the form of statistical visualizations, which are relatively difficult for many SME practitioners to interpret due to limited data literacy. This study aims to develop a web-based sentiment analysis chatbot for SMEs that is capable of presenting analysis results interactively and in an easily understandable manner through a natural language approach by integrating Retrieval-Augmented Generation (RAG). The research method employed is a system development approach using the Fountain Model, which supports iterative and parallel processes, including requirements analysis, specification, design, implementation, as well as system testing and integration. The backend system was developed using the NestJS framework with a modular architecture, REST APIs secured with JSON Web Token (JWT), and management of Instagram social media sentiment analysis data in JSON format. RAG integration was implemented using LangChain.js by leveraging embedding models and Large Language Models (LLMs), with vector storage in PostgreSQL utilizing the \textit{pgvector} extension to support semantic similarity-based retrieval. System testing was conducted using the \textit{Black Box Testing} method to evaluate REST API functionality and chatbot response relevance. The results show that all functional REST API testing scenarios were successfully executed, and the chatbot was able to generate contextual, relevant, and fact-based responses from the sentiment analysis knowledge base. Therefore, the proposed RAG-based chatbot system is proven to enhance the accessibility and utilization of sentiment analysis results for SMEs and has the potential to be further developed through the implementation of dynamic parameters and conversational memory to support more adaptive multi-turn interactions.}\\[0.6cm]
%% edit sampai sini

%% DILARANG EDIT BAGIAN INI
\noindent \textit{Key words: sentiment analysis, chatbot, Retrieval-Augmented Generation, SMEs, NestJS.}
%% DILARANG EDIT BAGIAN INI