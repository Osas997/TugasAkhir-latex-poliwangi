%==================================================================
% Ini adalah bab 1
% Silahkan edit sesuai kebutuhan, baik menambah atau mengurangi \section, \subsection
%==================================================================

\chapter[PENDAHULUAN]{\\ PENDAHULUAN}

\section{Latar Belakang}
Analisis sentimen terhadap komentar, ulasan, atau unggahan di media sosial dapat memberikan wawasan berharga mengenai persepsi publik terhadap suatu usaha mikro, kecil, dan menengah (UMKM). Misalnya, \citep{permana_sentimen_2023} melakukan analisis sentimen terhadap opini masyarakat mengenai UMKM di \textit{Twitter} dan menemukan bahwa sebagian besar sentimen bersifat positif (58\%), diikuti netral (37\%) dan negatif (4\%). Temuan seperti ini dapat menjadi dasar pengambilan keputusan strategis dalam pemasaran digital dan pengembangan produk.

Namun, hasil analisis semacam ini umumnya disajikan dalam bentuk visualisasi data seperti grafik batang, diagram lingkaran, atau grafik garis. Sayangnya, tidak semua pengguna mampu memahami makna dari grafik tersebut. Penelitian oleh \citep{quadri_you_2024} mengungkap bahwa tingkat pemahaman audiens terhadap grafik sangat bervariasi dan sering kali tidak sejalan dengan tujuan pembuat visualisasi. Hal serupa disampaikan oleh \citep{knoll_gulf_2025}, yang menunjukkan adanya \textit{gulf of interpretation} antara penyaji data dan pembaca, terutama ketika grafik memiliki kompleksitas tinggi. Akibatnya, sebagian pengguna kesulitan menafsirkan \textit{insight} yang sebenarnya ingin disampaikan melalui visualisasi data.

Sementara itu, pelaku UMKM di Indonesia semakin aktif memanfaatkan media sosial seperti \textit{Instagram} untuk promosi dan interaksi dengan pelanggan. Laporan \citep{datareportal_digital_2025} mencatat bahwa lebih dari 80\% pengguna internet di Indonesia aktif menggunakan \textit{Instagram}, menjadikannya sumber data potensial dalam memahami sentimen publik. Namun, pelaku UMKM yang tidak \textit{familiar} dengan statistik dan visualisasi data sering kali kesulitan menafsirkan hasil analisis tersebut secara mandiri.

Untuk mengatasi permasalahan ini, diperlukan sistem yang mampu menerjemahkan hasil analisis data menjadi informasi yang mudah dipahami melalui interaksi berbasis bahasa alami. Salah satu pendekatan modern yang relevan adalah \textit{Retrieval-Augmented Generation} (RAG), yaitu metode yang menggabungkan proses \textit{retrieval} (pengambilan informasi dari \textit{knowledge base}) dengan \textit{generation} (pembuatan jawaban menggunakan model bahasa). Pendekatan ini dikenal sebagai jembatan yang efektif antara data mentah dan pemahaman pengguna \citep{lewis_retrieval-augmented_2021, murtiyoso_systematic_2025, pokhrel_practical_2025}. RAG memungkinkan \textit{chatbot} memberikan jawaban yang lebih akurat, kontekstual, dan sesuai dengan data sebenarnya dibandingkan model generatif biasa. Berbeda dengan penggunaan model bahasa generatif biasa yang hanya mengandalkan pengetahuan statis hasil pelatihan, sistem chatbot dalam penelitian ini dituntut untuk memberikan jawaban berdasarkan data sentimen yang bersifat dinamis dan terus berubah seiring waktu. Komentar Instagram pada akun UMKM dapat bertambah setiap hari, sehingga distribusi sentimen, faktor positif dan negatif, serta pola interaksi pengguna selalu mengalami pembaruan. Jika hanya menggunakan model AI generatif tanpa mekanisme pengambilan data aktual, maka jawaban yang dihasilkan berpotensi tidak mencerminkan kondisi terbaru dari sentimen publik terhadap UMKM tersebut. Oleh karena itu, diperlukan pendekatan yang memungkinkan chatbot selalu merujuk pada hasil analisis sentimen terkini yang tersimpan dalam \textit{knowledge base} sistem. Dalam konteks inilah, \textit{Retrieval-Augmented Generation} (RAG) menjadi krusial karena memungkinkan model menghasilkan jawaban yang tidak hanya kontekstual, tetapi juga berbasis data aktual yang terus diperbarui.


Dari sisi implementasi teknologi, penggunaan NestJS dipilih karena memiliki arsitektur modular, dukungan penuh terhadap \textit{TypeScript}, serta kemampuan untuk membangun \textit{backend} yang terstruktur, terukur, dan mudah dipelihara \citep{muhammad_development_2024}. NestJS merupakan \textit{framework backend} berbasis Node.js yang menerapkan konsep \textit{server-side application framework} dengan pendekatan arsitektur yang terinspirasi dari \textit{Angular}, seperti penggunaan \textit{module}, \textit{controller}, dan \textit{service}. Pendekatan ini membantu pengembang dalam memisahkan tanggung jawab setiap komponen sistem secara jelas, sehingga alur pengembangan dan pengujian menjadi lebih terorganisir. Selain itu, NestJS menyediakan mekanisme \textit{dependency injection} yang memudahkan pengelolaan dependensi antar komponen serta meningkatkan keterbacaan dan \textit{maintainability} kode.

NestJS juga memungkinkan pengembangan yang fleksibel untuk mengintegrasikan berbagai layanan seperti layanan analisis sentimen berbasis aspek dan rekomendasi konten, sehingga sangat cocok digunakan dalam sistem berskala menengah hingga besar. Dengan dukungan \textit{middleware}, \textit{guard}, \textit{interceptor}, dan \textit{exception filter}, NestJS mampu menangani kebutuhan keamanan, validasi, serta manajemen \textit{request-response} secara konsisten pada \textit{layer backend}. Untuk integrasi RAG, digunakan LangChain.js, yaitu \textit{framework} modern yang dirancang untuk mempermudah pengelolaan \textit{pipeline retrieval}, integrasi \textit{embedding}, serta pembuatan rantai proses (\textit{chains}) pada sistem AI generatif \citep{vidivelli_efficiency-driven_2024}.

Berdasarkan kebutuhan tersebut, penelitian ini berfokus pada pengembangan \textit{chatbot} analisis sentimen UMKM berbasis \textit{web} dengan integrasi RAG menggunakan LangChain.js. Sistem ini memanfaatkan data hasil \textit{scraping} \textit{Instagram} yang diolah menjadi kategori sentimen (positif, netral, negatif) dan disimpan sebagai file JSON. File ini berfungsi ganda: sebagai sumber visualisasi di \textit{frontend} dan sebagai \textit{knowledge base} untuk \textit{chatbot} berbasis RAG.

Dengan demikian, penelitian ini tidak hanya membangun sistem analisis, tetapi juga berupaya menjembatani kesenjangan antara data kuantitatif dan pemahaman pengguna melalui interaksi berbasis bahasa alami, sehingga dapat membantu pelaku UMKM memperoleh \textit{insight} tanpa harus membaca grafik atau laporan statistik yang kompleks

\section{Perumusan Masalah}
Berdasarkan uraian latar belakang di atas, maka perumusan masalah dalam penelitian ini adalah sebagai berikut:
\begin{enumerate}
  \item Bagaimana merancang dan mengimplementasikan API \textit{backend} berbasis NestJS yang menyediakan layanan autentikasi serta \textit{endpoint chatbot} analisis sentimen UMKM?
  \item Bagaimana mengintegrasikan LangChain.js untuk metode RAG dengan \textit{knowledge base} berbentuk file JSON hasil analisis sentimen media sosial UMKM, agar \textit{chatbot} dapat menghasilkan jawaban yang relevan dan kontekstual terhadap pertanyaan pengguna?
\end{enumerate}

\section{Tujuan}
Tujuan dari penelitian ini adalah:
\begin{enumerate}
  \item Mengembangkan API \textit{backend} berbasis NestJS yang mencakup modul autentikasi JWT dan layanan \textit{chatbot} analisis sentimen UMKM.
  \item Mengimplementasikan metode RAG menggunakan LangChain.js dengan \textit{knowledge base} berbentuk file JSON hasil analisis sentimen.
\end{enumerate}

\section{Manfaat}
Penelitian ini diharapkan memberikan manfaat sebagai berikut:

\subsection{Manfaat Teoritis}
\begin{enumerate}
  \item Memberikan kontribusi pada literatur mengenai penerapan metode \textit{Retrieval-Augmented Generation} (RAG) dalam \textit{chatbot} berbasis \textit{web} berbahasa Indonesia.
  \item Menjadi referensi bagi penelitian selanjutnya di bidang integrasi \textit{natural language generation} dengan analisis data visual UMKM.
\end{enumerate}

\subsection{Manfaat Praktis}
\begin{enumerate}
  \item Bagi Pelaku UMKM: Memberikan alat bantu interaktif untuk memahami hasil analisis sentimen publik terhadap produk/layanan mereka.
  \item Bagi Pengembang Sistem: Memberikan \textit{blueprint} teknis (NestJS + LangChain.js + JSON \textit{knowledge base}) yang dapat diterapkan pada proyek serupa.
  \item Bagi Lembaga atau Pemerintah: Menjadi dasar pengembangan sistem \textit{chatbot} berbasis AI untuk meningkatkan literasi data dan strategi pemasaran digital UMKM.
\end{enumerate}

\section{Batasan Masalah}

\begin{enumerate}
  \item Data yang digunakan berasal dari hasil \textit{scraping} akun \textit{Instagram} UMKM yang sudah sukses (misalnya Mie Gacoan dan Sambal Bakar) di wilayah Jawa Timur.
  \item Fokus utama penelitian adalah pada pengembangan sistem \textit{backend} berbasis NestJS dan implementasi \textit{chatbot} berbasis \textit{Retrieval-Augmented Generation} (RAG) menggunakan LangChain.js untuk interpretasi hasil analisis sentimen.
  \item Penelitian ini tidak mencakup proses \textit{scraping} data \textit{Instagram}, \textit{preprocessing} data mentah, atau pengembangan model analisis sentimen, karena komponen-komponen tersebut dikembangkan oleh tim kolaborasi terpisah.
  \item Evaluasi penelitian berfokus pada aspek teknis implementasi sistem (fungsionalitas API, akurasi \textit{retrieval}, relevansi jawaban \textit{chatbot}) dan tidak mencakup evaluasi pengalaman pengguna (\textit{user experience testing} atau \textit{user acceptance testing}).
  \item Penelitian ini tidak melakukan pengujian metrik akurasi klasifikasi seperti precision, recall, F1-score, atau confusion matrix terhadap output chatbot, karena fokus evaluasi terletak pada fungsionalitas sistem dan relevansi jawaban berbasis knowledge base, bukan pada performa klasifikasi model machine learning.
  \item \textit{Knowledge base} untuk sistem RAG terbatas pada data hasil analisis sentimen dalam format JSON dan tidak mencakup sumber pengetahuan eksternal lainnya.
\end{enumerate}

% \section{Keaslian Gagasan}
% Penelitian ini menawarkan pendekatan inovatif dalam membantu pelaku UMKM memahami hasil analisis sentimen melalui integrasi teknologi Retrieval-Augmented Generation (RAG) pada chatbot berbasis web. Keaslian gagasan dalam penelitian ini terletak pada beberapa aspek berikut:

% \begin{enumerate}
%   \item Integrasi Analisis Sentimen dengan Chatbot Interaktif Berbasis RAG
%         \begin{itemize}
%           \item Berbeda dengan penelitian terdahulu seperti \citep{permana_sentimen_2023} dan \citep{zaenab_kurnia_analisis_2024} yang hanya menyajikan hasil analisis sentimen dalam bentuk klasifikasi dan visualisasi grafik
%           \item Penelitian ini mengintegrasikan hasil analisis sentimen dengan sistem chatbot interaktif yang mampu memberikan interpretasi kontekstual
%           \item Menyediakan mekanisme tanya-jawab berbasis bahasa alami untuk pengguna yang kurang familiar dengan statistik
%         \end{itemize}

%   \item Kombinasi Teknologi Modern: NestJS dan LangChain.js
%         \begin{itemize}
%           \item Menggunakan kombinasi NestJS sebagai backend framework dan LangChain.js untuk orkestrasi RAG pipeline
%           \item \citep{muhammad_development_2024} menunjukkan keunggulan NestJS dalam pembangunan REST API terstruktur, namun belum mengintegrasikannya dengan teknologi AI generatif
%           \item Penelitian ini mengisi celah tersebut dengan memberikan blueprint implementasi teknis yang dapat direplikasi untuk proyek serupa
%         \end{itemize}

%   \item Solusi untuk Kesenjangan Pemahaman Visualisasi Data (Gulf of Interpretation)
%         \begin{itemize}
%           \item Menjawab masalah yang diidentifikasi oleh \citep{knoll_gulf_2025} dan \citep{quadri_you_2024} tentang kesenjangan pemahaman antara penyaji data dan pembaca
%           \item Menyediakan antarmuka berbahasa alami untuk menjembatani data kuantitatif dan pemahaman pengguna non-teknis
%           \item Berbeda dengan chatbot konvensional yang memberikan respons statis, sistem ini menghasilkan jawaban kontekstual berdasarkan data sentimen aktual
%         \end{itemize}

% \end{enumerate}

% \section{Sistematika Penulisan}
% Laporan tugas akhir ini disusun secara sistematis dalam beberapa bab untuk memudahkan pembaca dalam memahami alur penelitian dari awal hingga akhir. Sistematika penulisan laporan ini adalah sebagai berikut:

% \textbf{BAB 1: PENDAHULUAN}

% Bab ini membahas latar belakang masalah yang melatarbelakangi penelitian, yaitu kesulitan pelaku UMKM dalam memahami hasil analisis sentimen yang disajikan dalam bentuk visualisasi data. Bab ini juga mencakup perumusan masalah, tujuan penelitian, manfaat penelitian baik secara teoritis maupun praktis, batasan masalah penelitian, keaslian gagasan yang menunjukkan keunikan dan kontribusi penelitian, serta sistematika penulisan laporan.

% \textbf{BAB 2: TINJAUAN PUSTAKA}

% Bab ini menguraikan landasan teori dan konsep-konsep yang relevan dengan penelitian, meliputi: konsep Usaha Mikro, Kecil, dan Menengah (UMKM), peran media sosial sebagai platform pemasaran, Natural Language Processing (NLP), analisis sentimen, sistem rekomendasi, serta kajian penelitian terkait dari tahun 2021-2025 yang membahas topik serupa. Tinjauan pustaka ini menjadi fondasi teoritis dalam pengembangan sistem.

% \textbf{BAB 3: METODOLOGI PENELITIAN}

% Bab ini menjelaskan metode penelitian yang digunakan, desain sistem yang akan dikembangkan, karakteristik dataset yang digunakan (hasil scraping Instagram UMKM), metode analisis sentimen yang diterapkan, metode sistem rekomendasi (jika ada), evaluasi sistem yang akan dilakukan, serta jadwal penelitian. Bab ini memberikan gambaran komprehensif mengenai tahapan dan proses pengembangan sistem chatbot berbasis RAG.

% \textbf{BAB 4: HASIL DAN PEMBAHASAN (akan dikembangkan)}

% Bab ini akan menyajikan hasil implementasi sistem, meliputi: implementasi API backend menggunakan NestJS, integrasi LangChain.js untuk RAG pipeline, mekanisme retrieval dan generation, serta hasil uji coba sistem. Pembahasan akan mencakup analisis terhadap kinerja sistem, kemampuan chatbot dalam menjawab pertanyaan pengguna, dan evaluasi berdasarkan uji coba pengguna.

% \textbf{BAB 5: PENUTUP (akan dikembangkan)}

% Bab ini berisi kesimpulan dari keseluruhan penelitian yang menjawab rumusan masalah dan tujuan penelitian, serta saran untuk pengembangan lebih lanjut, baik dari sisi teknis maupun fungsional sistem.
