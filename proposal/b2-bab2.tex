%==================================================================
% Ini adalah bab 2
% Silahkan edit sesuai kebutuhan, baik menambah atau mengurangi \section, \subsection
%==================================================================

\chapter[TINJAUAN PUSTAKA]{\\ TINJAUAN PUSTAKA}

\section{Usaha Mikro, Kecil, dan Menengah (UMKM)}
Usaha Mikro, Kecil, dan Menengah (UMKM) merupakan salah satu pilar utama perekonomian Indonesia yang memiliki peran strategis dalam pertumbuhan ekonomi nasional, penyerapan tenaga kerja, dan pengurangan kemiskinan. UMKM diklasifikasikan berdasarkan skala usaha, nilai aset, dan omzet tahunan yang dimiliki. Sektor UMKM memberikan kontribusi signifikan terhadap Produk Domestik Bruto (PDB) Indonesia dan menyerap sebagian besar tenaga kerja nasional, menjadikannya tulang punggung perekonomian Indonesia.

UMKM memiliki karakteristik yang unik, antara lain: fleksibilitas dalam beradaptasi dengan perubahan pasar, kemampuan memanfaatkan sumber daya lokal, serta kedekatan dengan konsumen. Namun, UMKM juga menghadapi berbagai tantangan seperti keterbatasan modal, akses teknologi yang terbatas, kemampuan manajerial yang kurang, serta kesulitan dalam pemasaran dan branding. Di era digital saat ini, UMKM dituntut untuk dapat beradaptasi dengan teknologi informasi dan memanfaatkan platform digital untuk meningkatkan daya saing mereka.

Dalam konteks pemasaran digital, UMKM perlu memahami persepsi dan sentimen konsumen terhadap produk atau layanan mereka. Analisis sentimen dari media sosial dapat memberikan wawasan berharga tentang bagaimana konsumen merespons brand, produk, atau kampanye pemasaran yang dilakukan \cite{permana_sentimen_2023}. Namun, banyak pelaku UMKM yang belum memiliki kemampuan atau sumber daya untuk melakukan analisis data secara mendalam, sehingga diperlukan tools yang mudah digunakan dan dapat memberikan insight yang actionable.

\section{Media Sosial sebagai Platform Pemasaran}
Media sosial telah menjadi salah satu platform pemasaran yang paling efektif di era digital, terutama bagi UMKM yang memiliki keterbatasan budget untuk iklan konvensional. Platform seperti Instagram, Facebook, TikTok, dan Twitter menawarkan berbagai fitur yang memungkinkan pelaku usaha untuk berinteraksi langsung dengan konsumen, membangun brand awareness, serta melakukan promosi dengan biaya yang relatif terjangkau.

Instagram, sebagai salah satu platform media sosial yang paling populer di Indonesia, memiliki lebih dari 80\% penetrasi di kalangan pengguna internet Indonesia menurut laporan \cite{datareportal_digital_2025}. Platform ini menyediakan berbagai fitur seperti Instagram Feed, Stories, Reels, dan Shopping yang dapat dimanfaatkan oleh UMKM untuk mempromosikan produk mereka. Fitur komentar dan caption pada postingan Instagram menjadi sumber data yang kaya untuk memahami sentimen dan opini konsumen terhadap suatu brand atau produk.

\cite{utari_strategi_2022} menyatakan bahwa strategi pemasaran UMKM melalui Instagram di era pemasaran 4.0 harus memperhatikan beberapa aspek penting seperti konten visual yang menarik, storytelling yang autentik, interaksi aktif dengan followers, serta pemanfaatan fitur-fitur Instagram untuk meningkatkan engagement. Lebih lanjut, \cite{arviani_sosial_2021} menjelaskan bahwa social media marketing memberikan peluang sekaligus tantangan bagi UMKM lokal, terutama dalam masa pandemi COVID-19 di mana aktivitas bisnis banyak beralih ke platform digital.

Komentar, ulasan, dan mentions di media sosial merupakan bentuk Electronic Word of Mouth (eWOM) yang memiliki pengaruh signifikan terhadap keputusan pembelian konsumen. \cite{fitriani_tweeting_2023} menunjukkan bahwa sentimen di media sosial dapat mencerminkan kondisi ekonomi makro dan persepsi masyarakat terhadap berbagai isu. Oleh karena itu, analisis sentimen dari data media sosial menjadi penting untuk membantu UMKM memahami bagaimana konsumen mempersepsikan brand mereka dan mengidentifikasi area yang perlu diperbaiki.

\section{Natural Language Processing (NLP)}
Natural Language Processing (NLP) adalah cabang dari artificial intelligence yang berfokus pada interaksi antara komputer dan bahasa manusia. NLP memungkinkan komputer untuk memahami, menginterpretasikan, dan menghasilkan bahasa manusia dalam bentuk yang bermakna dan berguna. Teknologi NLP telah berkembang pesat dalam beberapa tahun terakhir, didorong oleh kemajuan dalam deep learning dan ketersediaan data teks yang besar.

Beberapa tugas utama dalam NLP meliputi: tokenization (pemecahan teks menjadi unit-unit kecil seperti kata atau kalimat), part-of-speech tagging (penandaan kelas kata), named entity recognition (identifikasi entitas seperti nama orang, tempat, atau organisasi), sentiment analysis (analisis sentimen), machine translation (terjemahan mesin), text summarization (peringkasan teks), dan question answering (sistem tanya jawab).

Dalam konteks penelitian ini, NLP berperan penting dalam beberapa aspek:
\begin{enumerate}
  \item \textbf{Preprocessing teks}: Membersihkan dan mempersiapkan data teks dari media sosial yang sering kali mengandung noise seperti emoji, hashtag, mention, dan bahasa informal.
  \item \textbf{Analisis sentimen}: Mengklasifikasikan sentimen dari komentar atau caption menjadi kategori positif, netral, atau negatif.
  \item \textbf{Retrieval}: Mencari dan mengambil informasi yang relevan dari knowledge base berdasarkan query pengguna.
  \item \textbf{Generation}: Menghasilkan jawaban yang koheren dan kontekstual dalam bahasa natural menggunakan model generatif.
\end{enumerate}

Perkembangan terkini dalam NLP ditandai dengan munculnya Large Language Models (LLM) seperti GPT, BERT, dan model-model berbasis transformer lainnya yang mampu memahami konteks bahasa dengan lebih baik dan menghasilkan teks yang lebih natural. Model-model ini menjadi dasar bagi berbagai aplikasi NLP modern, termasuk chatbot berbasis AI generatif.

\section{Analisis Sentimen}
Analisis sentimen, juga dikenal sebagai opinion mining, adalah proses komputasional untuk mengidentifikasi dan mengekstrak informasi subjektif dari teks, seperti pendapat, emosi, dan sikap penulis terhadap suatu topik, produk, atau layanan. Analisis sentimen umumnya mengklasifikasikan teks menjadi tiga kategori utama: positif, negatif, dan netral, meskipun beberapa pendekatan juga menggunakan klasifikasi yang lebih granular seperti sangat positif, positif, netral, negatif, dan sangat negatif.

\subsection{Pendekatan Analisis Sentimen}
Terdapat beberapa pendekatan utama dalam analisis sentimen:
\begin{enumerate}
  \item \textbf{Lexicon-based approach}: Menggunakan kamus sentimen yang berisi daftar kata-kata beserta polaritas sentimen mereka. Pendekatan ini relatif sederhana namun kurang akurat untuk teks yang kompleks atau mengandung sarkasme.
  \item \textbf{Machine learning approach}: Menggunakan algoritma machine learning seperti Naive Bayes, Support Vector Machine (SVM), atau Random Forest yang dilatih dengan data berlabel untuk mengklasifikasikan sentimen. \cite{permana_sentimen_2023} menggunakan Naive Bayes Classifier untuk analisis sentimen opini masyarakat terhadap UMKM di Twitter.
  \item \textbf{Deep learning approach}: Menggunakan neural networks seperti LSTM, CNN, atau transformer-based models (BERT, RoBERTa) yang mampu menangkap konteks dan semantic yang lebih kompleks.\cite{ningrum_pengembangan_2025} menggunakan BERT untuk analisis sentimen komentar TikTok UMKM.
\end{enumerate}

\subsection{Analisis Sentimen pada Media Sosial}
Analisis sentimen pada media sosial memiliki tantangan khusus karena karakteristik teks di media sosial yang informal, mengandung slang, singkatan, emoji, dan sering kali tidak mengikuti tata bahasa yang baku.\cite{zaenab_kurnia_analisis_2024} dalam penelitiannya tentang analisis sentimen komentar kerja sama TikTok Shop dan Tokopedia di Instagram menggunakan Naive Bayes Classifier dan memperoleh akurasi 84\%.

\cite{ibrahim_sentiment_2022} melakukan analisis sentimen terhadap Twitter dan Instagram Perpustakaan Nasional Indonesia dan menemukan bahwa mayoritas sentimen bersifat positif, yang menunjukkan kepuasan pengguna terhadap layanan perpustakaan. Studi ini menunjukkan bahwa analisis sentimen dapat memberikan feedback yang berharga untuk evaluasi dan perbaikan layanan.

\subsection{Aplikasi Analisis Sentimen untuk UMKM}
Bagi UMKM, analisis sentimen dapat digunakan untuk berbagai tujuan:

\begin{enumerate}
  \item \textbf{Memahami persepsi konsumen}: Mengetahui bagaimana konsumen mempersepsikan produk atau brand mereka
  \item \textbf{Identifikasi keluhan dan masalah}: Mendeteksi keluhan konsumen dengan cepat sehingga dapat ditangani segera.
  \item \textbf{Analisis kompetitor}: Membandingkan sentimen terhadap brand mereka dengan kompetitor. \cite{saffania_zahra_aina_halinda_pengaruh_2025} menggunakan analisis sentimen untuk membandingkan posisi bersaing Janji Jiwa dan Kopi Kenangan.
\end{enumerate}

Namun, hasil analisis sentimen yang berupa angka statistik atau visualisasi grafik sering kali sulit dipahami oleh pelaku UMKM yang tidak memiliki latar belakang teknis atau statistik. Oleh karena itu, diperlukan cara penyampaian insight yang lebih intuitif dan interaktif.

\section{Retrieval-Augmented Generation (RAG)}
Retrieval-Augmented Generation (RAG) adalah pendekatan dalam Natural Language Processing yang menggabungkan dua komponen utama: sistem retrieval (pengambilan informasi) dan model generatif (pembangkitan teks). Konsep RAG pertama kali diperkenalkan oleh \cite{lewis_retrieval-augmented_2021} sebagai solusi untuk meningkatkan akurasi dan relevansi jawaban dalam tugas-tugas NLP yang membutuhkan pengetahuan intensif (knowledge-intensive tasks).

\subsection{Arsitektur RAG}
\begin{figure}[h]
  \centering
  \includegraphics[width=0.8\textwidth]{rag-arsitektur.jpg}
  \caption{Arsitektur Retrieval-Augmented Generation (RAG)}
  \label{fig:rag-architecture}
\end{figure}

Arsitektur RAG terdiri dari dua pipeline utama yang bekerja secara terintegrasi: RAG Ingestion Pipeline dan Prompt Elements Pipeline, sebagaimana ditunjukkan pada Gambar \ref{fig:rag-architecture}.

RAG Ingestion Pipeline merupakan tahap persiapan knowledge base yang terdiri dari beberapa langkah:
\begin{enumerate}
  \item \textbf{Clean}: Data sources (dokumen, file JSON, atau text data) dibersihkan dari noise dan diformat secara konsisten untuk memastikan kualitas data yang akan diproses.
  \item \textbf{Chunk}: Cleaned documents dipecah menjadi chunks atau potongan-potongan kecil yang lebih mudah diproses. Chunking ini penting karena embedding models memiliki batasan panjang input, dan chunks yang lebih kecil memungkinkan retrieval yang lebih presisi.
  \item \textbf{Embed}: Setiap chunked document diubah menjadi embedding vectors menggunakan embedding model. Embedding adalah representasi numerical dari teks dalam bentuk vector berdimensi tinggi yang menangkap semantic meaning dari konten.
  \item \textbf{Vector DB}: Embedded chunks beserta metadata-nya disimpan dalam vector database yang memungkinkan pencarian berdasarkan similarity. Vector database seperti Pinecone, Chroma, atau FAISS dioptimalkan untuk melakukan similarity search dengan cepat pada jutaan vectors.
\end{enumerate}

Prompt Elements Pipeline merupakan tahap query dan generation yang melibatkan:

\begin{enumerate}
  \item \textbf{User Input}: Pertanyaan atau query dari pengguna diterima oleh sistem.
  \item \textbf{User Input Embedding}: Query pengguna diubah menjadi embedding vector menggunakan embedding model yang sama dengan yang digunakan pada ingestion pipeline. Konsistensi embedding model ini penting untuk memastikan similarity calculation yang akurat.
  \item \textbf{Vector DB Retrieval}: Embedding query digunakan untuk mencari xK (top-K) documents yang paling relevan dari vector database berdasarkan cosine similarity atau distance metrics lainnya. Retrieved chunks ini disertai dengan metadata-nya yang dapat berisi informasi seperti source, timestamp, atau kategori.
  \item \textbf{Prompt Construction}: Retrieved documents (xK retrieved chunks) digabungkan dengan user input dan prompt template untuk membentuk prompt yang lengkap. Prompt template mendefinisikan struktur dan instruksi bagaimana LLM harus menggunakan context yang diberikan untuk generate jawaban.
  \item \textbf{LLM Generation}: Prompt yang telah dikonstruksi dikirim ke Large Language Model (LLM) yang akan menghasilkan jawaban berdasarkan context dari retrieved documents dan pertanyaan pengguna.
  \item \textbf{Generated Answer}: Output dari LLM berupa jawaban dalam bahasa natural yang relevan dengan pertanyaan pengguna dan grounded pada informasi yang ada di knowledge base.
\end{enumerate}

Arsitektur RAG ini memiliki keunggulan dibandingkan dengan pure generative model karena menggabungkan kekuatan retrieval (akses ke knowledge base yang luas dan dapat diperbarui) dengan generation (kemampuan LLM untuk menghasilkan jawaban yang natural dan kontekstual). Proses retrieval memastikan bahwa jawaban didasarkan pada informasi faktual dari knowledge base, sementara LLM memberikan kemampuan untuk menyajikan informasi tersebut dalam format yang mudah dipahami dan sesuai dengan konteks pertanyaan pengguna.

\subsection{Keunggulan RAG}
RAG memiliki beberapa keunggulan dibandingkan dengan model generatif biasa:

\begin{enumerate}
  \item \textbf{Akurasi lebih tinggi}: Dengan memanfaatkan knowledge base eksternal, RAG dapat memberikan jawaban yang lebih akurat dan berdasarkan fakta, bukan hanya mengandalkan pengetahuan yang "diingat" oleh model dari proses training.
  \item \textbf{Kemampuan update pengetahuan}: Knowledge base dapat diperbarui tanpa perlu melatih ulang model generatif, sehingga sistem dapat tetap up-to-date dengan informasi terbaru.
  \item \textbf{Transparansi dan verifikasi}: Sumber informasi yang digunakan untuk menghasilkan jawaban dapat diidentifikasi dan diverifikasi, meningkatkan trustworthiness sistem.
  \item \textbf{Efisiensi}: Tidak perlu menyimpan semua pengetahuan di dalam parameter model, sehingga lebih efisien dari segi komputasi dan storage.
\end{enumerate}

\subsection{Implementasi RAG dalam Berbagai Domain}
\cite{murtiyoso_systematic_2025} melakukan systematic review tentang penggunaan RAG untuk meningkatkan domain-specific knowledge dalam Large Language Models dan menemukan bahwa RAG efektif untuk berbagai aplikasi seperti question answering, dialog systems, dan information extraction.

Beberapa implementasi RAG dalam berbagai domain meliputi:

\begin{enumerate}
  \item \textbf{Helpdesk dan customer service}: \cite{pratama_pengembangan_2024} mengembangkan sistem helpdesk menggunakan chatbot dengan metode RAG yang mampu memberikan respons yang relevan terhadap pertanyaan pengguna dengan memanfaatkan dokumen helpdesk sebagai knowledge base.
  \item \textbf{Layanan akademik}: \cite{husain_development_2025} mengembangkan chatbot layanan akademik berbasis RAG yang mampu menjawab pertanyaan mahasiswa tentang layanan akademik dengan akurat menggunakan dokumen akademik sebagai knowledge base.
  \item \textbf{Domain hukum}: \cite{pratama_pengembangan_2024} menerapkan RAG untuk informasi hukum pidana Indonesia menggunakan model LLaMA, menunjukkan bahwa RAG dapat digunakan untuk domain-specific knowledge base dalam bahasa Indonesia.
  \item \textbf{Website-based chatbot}: \cite{pokhrel_practical_2025} mengimplementasikan RAG untuk chatbot berbasis website dengan menggabungkan web scraping, vektorisasi, dan semantic search, memberikan panduan praktis tentang implementasi RAG.
  \item \textbf{Document question answering}: \cite{muludi_retrieval-augmented_2024} menggunakan pendekatan RAG untuk document question answering menggunakan large language model, menunjukkan bahwa RAG dapat meningkatkan akurasi jawaban dengan memanfaatkan dokumen sebagai konteks tambahan.
\end{enumerate}

\subsection{LangChain sebagai Framework RAG}
LangChain adalah framework open-source yang mempermudah pengembangan aplikasi berbasis Large Language Models (LLM). LangChain menyediakan abstraksi dan tools yang memudahkan developer dalam membangun RAG pipeline, termasuk document loaders, text splitters, vector stores, retrievers, dan chains untuk mengorkestrasi alur kerja yang kompleks.

\cite{vidivelli_efficiency-driven_2024} menunjukkan bahwa penggunaan LangChain dalam pengembangan custom chatbot dapat meningkatkan efisiensi development dan memungkinkan integrasi yang lebih mudah dengan berbagai model LLM dan vector databases. LangChain juga tersedia dalam versi JavaScript (LangChain.js) yang memungkinkan implementasi RAG dalam ekosistem Node.js.

\section{Chatbot dan Interaksi Bahasa Natural}
Chatbot adalah program komputer yang dirancang untuk mensimulasikan percakapan dengan manusia melalui text atau voice interface. Dalam konteks modern, chatbot telah berkembang dari rule-based systems yang sederhana menjadi AI-powered chatbot yang mampu memahami konteks dan memberikan respons yang lebih natural dan relevan.

\subsection{Jenis-jenis Chatbot}
Terdapat beberapa jenis chatbot berdasarkan kompleksitas dan teknologi yang digunakan:

\begin{enumerate}
  \item \textbf{Rule-based chatbot}: Bekerja berdasarkan aturan yang telah ditentukan sebelumnya (if-then rules). Chatbot jenis ini terbatas pada skenario yang telah diprogram dan tidak dapat menangani pertanyaan di luar aturan yang ada.
  \item \textbf{Retrieval-based chatbot}: Memilih respons terbaik dari sekumpulan respons yang telah ditentukan sebelumnya berdasarkan similarity dengan input pengguna.
  \item \textbf{Generative chatbot}: Menghasilkan respons baru untuk setiap input menggunakan model bahasa. Chatbot jenis ini lebih fleksibel namun berisiko menghasilkan respons yang tidak akurat atau tidak relevan.
  \item \textbf{Hybrid chatbot}: Menggabungkan pendekatan retrieval dan generative, seperti dalam metode RAG, untuk mendapatkan keunggulan dari kedua pendekatan tersebut.
\end{enumerate}

\subsection{Chatbot untuk Visualisasi dan Interpretasi Data}
\cite{kavaz_chatbot-based_2023} melakukan scoping review tentang chatbot berbasis natural language interfaces untuk visualisasi data dan mengidentifikasi bahwa chatbot dapat menjadi antarmuka yang efektif untuk membantu pengguna memahami visualisasi data kompleks, terutama bagi pengguna yang tidak familiar dengan statistik. Chatbot dapat membantu pengguna dengan:

\begin{enumerate}
  \item Menjawab pertanyaan tentang data yang divisualisasikan
  \item Menjelaskan insight atau pattern yang terdapat dalam grafik
  \item Memberikan rekomendasi berdasarkan data
  \item Membantu pengguna melakukan eksplorasi data interaktif
\end{enumerate}

Penelitian \cite{quadri_you_2024} menunjukkan bahwa tingkat pemahaman audiens terhadap grafik sangat bervariasi dan sering kali tidak sejalan dengan tujuan pembuat visualisasi. \cite{knoll_gulf_2025} juga mengidentifikasi adanya "gulf of interpretation" antara pembuat grafik dan pembaca. Chatbot berbasis bahasa natural dapat menjadi solusi untuk menjembatani kesenjangan pemahaman ini dengan memberikan penjelasan yang disesuaikan dengan level pemahaman pengguna.
\subsection{Chatbot untuk Business Intelligence}
\cite{salim_llm_2025} mengembangkan LLM-based QA chatbot builder menggunakan generative AI untuk question answering, menunjukkan bahwa chatbot dapat digunakan untuk menjawab pertanyaan bisnis berdasarkan data perusahaan. \cite{cempaka_influence_2025} meneliti pengaruh sistem informasi chatbot terhadap customer experience dan social media engagement di marketplace, menemukan bahwa chatbot dapat meningkatkan engagement dan kepuasan pelanggan.

Dalam konteks UMKM, chatbot dapat berfungsi sebagai "data analyst virtual" yang membantu pelaku usaha memahami data mereka tanpa perlu memiliki keahlian teknis dalam data analysis atau statistik. Chatbot dapat menjawab pertanyaan seperti "Bagaimana sentimen konsumen terhadap produk saya?", "Apa keluhan yang paling sering muncul?", atau "Bagaimana sentimen bulan ini dibandingkan bulan lalu?".

\section{NestJS sebagai Backend Framework}
NestJS adalah framework Node.js yang menerapkan arsitektur modular dan berbasis TypeScript. Menurut \cite{muhammad_development_2024}, NestJS cocok digunakan untuk membangun sistem backend berskala menengah hingga besar karena:

\begin{enumerate}
  \item Mendukung modularisasi dan dependency injection.
  \item Memiliki ekosistem yang besar dan aktif.
  \item Memiliki struktur proyek yang rapi dan mudah dipelihara.
  \item Mendukung integrasi API Gateway serta berbagai metode autentikasi seperti JWT.
\end{enumerate}

Dengan kemampuan tersebut, NestJS menjadi fondasi yang tepat untuk sistem chatbot analisis sentimen pada penelitian ini.

\section{Penelitian Terkait}
Berikut adalah tabel perbandingan penelitian terkait yang relevan dengan pengembangan chatbot analisis sentimen UMKM berbasis RAG:


\begin{longtable}{|c|p{2.8cm}|p{2.5cm}|p{4cm}|p{4.5cm}|}
  \caption{Tabel perbandingan penelitian terkait} \label{tab:perbandingan_penelitian}                                                                                                                                                               \\
  \hline
  \textbf{No} & \textbf{Peneliti}              & \textbf{Teknologi}                    & \textbf{Judul}                                                     & \textbf{Fitur}                                                                        \\ \hline
  \endfirsthead

  \hline
  \textbf{No} & \textbf{Peneliti}              & \textbf{Teknologi}                    & \textbf{Judul}                                                     & \textbf{Fitur}                                                                        \\ \hline
  \endhead

  \hline
  \endfoot

  1           & Permana et al. (2023)          & Naive Bayes Classifier, Twitter API   & Sentimen Analisis Opini Masyarakat Terhadap UMKM pada Twitter      & Klasifikasi sentimen (positif, netral, negatif) dengan akurasi tinggi untuk data UMKM \\ \hline
  2           & Zaenab Kurnia et al. (2024)    & Naive Bayes Classifier, Instagram API & Analisis Sentimen Komentar TikTok Shop \& Tokopedia di Instagram   & Analisis sentimen komentar Instagram dengan akurasi 84\%                              \\ \hline
  3           & Ningrum et al. (2025)          & BERT, TikTok API                      & Bot Komentar Otomatis dengan Analisis Sentimen BERT untuk UMKM     & Sentimen berbasis deep learning + auto comment untuk engagement                       \\ \hline
  4           & Lewis et al. (2021)            & RAG, BERT, DPR                        & Retrieval-Augmented Generation for Knowledge-Intensive NLP Tasks   & Paper dasar RAG: retrieval + generation untuk Q\&A                                    \\ \hline
  5           & Pratama \& Sisephaputra (2024) & RAG, LangChain, Vector DB             & Sistem Helpdesk Chatbot dengan Metode RAG                          & Retrieval dokumen + respons berbasis knowledge base                                   \\ \hline
  6           & Husain et al. (2025)           & RAG, LLM, Vector Store                & Academic Services Chatbot Based on RAG                             & Retrieval dokumen akademik + generation jawaban akurat                                \\ \hline
  7           & Pokhrel et al. (2025)          & RAG, Web Scraping, Semantic Search    & Practical Application of RAG for Website-Based Chatbots            & RAG untuk chatbot website: scraping + semantic search                                 \\ \hline
  8           & Vidivelli et al. (2024)        & LangChain, RAG, Custom LLM            & Efficiency-Driven Chatbot: LangChain + RAG + LLM Fusion            & Integrasi RAG efisien + optimasi performa                                             \\ \hline
  9           & Muhammad \& Paputungan (2024)  & NestJS, REST API, TypeScript          & Backend Server with REST API Architecture for E-Wallet System      & Modular backend NestJS + dependency injection + REST API                              \\ \hline
  10          & Kavaz et al. (2023)            & NLP, Chatbot, Data Visualization      & Natural Language Interface for Data Visualization (Scoping Review) & Review chatbot untuk interpretasi visualisasi data                                    \\ \hline
\end{longtable}


Berdasarkan tabel \ref{tab:perbandingan_penelitian}, dapat diamati bahwa penelitian terkait analisis sentimen UMKM dan implementasi RAG telah berkembang pesat dalam beberapa tahun terakhir. Penelitian 1--3 menunjukkan berbagai pendekatan analisis sentimen untuk UMKM di media sosial, mulai dari metode klasik Naive Bayes hingga deep learning berbasis BERT. Penelitian 4--8 mendemonstrasikan implementasi RAG dalam berbagai domain seperti helpdesk, layanan akademik, dan website, membuktikan efektivitas metode ini untuk meningkatkan akurasi dan relevansi respons chatbot. Penelitian 9 memberikan justifikasi penggunaan NestJS sebagai backend framework yang robust dan scalable. Penelitian 10 mendukung konsep chatbot sebagai solusi untuk mengatasi kesenjangan pemahaman visualisasi data.

\section{Analisis Gap Penelitian}

Berdasarkan tinjauan terhadap penelitian-penelitian terkait pada Tabel \ref{tab:perbandingan_penelitian}, dapat diidentifikasi beberapa gap atau celah penelitian yang menjadi peluang bagi penelitian ini untuk memberikan kontribusi yang unik dan bernilai. Analisis gap penelitian diuraikan sebagai berikut:

\subsection{Gap 1: Tidak Adanya Integrasi Analisis Sentimen dengan Chatbot Interaktif}
Penelitian-penelitian analisis sentimen UMKM seperti \cite{permana_sentimen_2023}, \cite{zaenab_kurnia_analisis_2024}, dan \cite{ningrum_pengembangan_2025} umumnya fokus pada pengembangan model klasifikasi dan penyajian hasil dalam bentuk visualisasi statis seperti grafik batang, diagram lingkaran, atau tabel confusion matrix. Tidak ada penelitian yang menyediakan mekanisme interaktif berbasis chatbot untuk membantu pengguna, khususnya pelaku UMKM non-teknis, dalam memahami dan menginterpretasikan hasil analisis sentimen tersebut.

Pelaku UMKM yang tidak memiliki latar belakang statistik atau data science sering kali kesulitan untuk mengekstrak actionable insights dari visualisasi data. Mereka mungkin dapat melihat bahwa 58\% sentimen adalah positif, tetapi tidak tahu apa implikasinya terhadap strategi pemasaran mereka, bagaimana membandingkan dengan periode sebelumnya, atau aspek mana yang perlu diperbaiki. Penelitian ini mengisi gap dengan mengembangkan chatbot berbasis RAG yang dapat memberikan interpretasi kontekstual terhadap hasil analisis sentimen melalui interaksi bahasa natural, memungkinkan pengguna untuk bertanya dan mendapatkan insight yang disesuaikan dengan kebutuhan mereka.

\subsection{Gap 2: Implementasi RAG Tidak Spesifik untuk Domain Analisis Sentimen UMKM}
Meskipun beberapa penelitian seperti \cite{pratama_pengembangan_2024}, \cite{husain_development_2025}, dan \cite{pratama_retrieval-augmented_2023} telah menerapkan RAG dalam berbagai domain seperti helpdesk, layanan akademik, dan hukum pidana, belum ada penelitian yang mengimplementasikan RAG spesifik untuk domain analisis sentimen UMKM di Indonesia. Penelitian-penelitian RAG yang ada umumnya menggunakan dokumen teks panjang (PDF, Word, atau web content) sebagai knowledge base, bukan data terstruktur hasil analisis.

Penelitian ini mengisi gap dengan mengadaptasi metode RAG menggunakan knowledge base berbentuk file JSON yang berisi hasil analisis sentimen dari media sosial (Instagram) UMKM. Data JSON ini memiliki struktur yang berbeda dari dokumen teks biasa, yaitu berisi informasi terstruktur seperti kategori sentimen, jumlah komentar per kategori, tren temporal, keywords yang sering muncul, dan contoh komentar. Pendekatan ini memerlukan strategi retrieval dan generation yang disesuaikan untuk dapat mengekstrak dan menyajikan informasi dari data terstruktur secara efektif.

\subsection{Gap 3: Kombinasi Teknologi Modern untuk Sistem Backend yang Terstruktur}
Penelitian \cite{muhammad_development_2024} menunjukkan keunggulan NestJS dalam pembangunan REST API yang terstruktur, modular, dan terukur untuk aplikasi e-wallet. NestJS menawarkan arsitektur yang solid dengan dependency injection, middleware support, dan integration dengan berbagai libraries. Namun, belum ada penelitian yang mengintegrasikan NestJS dengan framework AI generatif seperti LangChain.js untuk keperluan chatbot berbasis RAG.

LangChain.js adalah framework yang powerful untuk membangun aplikasi LLM, tetapi implementasinya dalam ekosistem Node.js dengan backend framework enterprise-grade seperti NestJS belum banyak dieksplorasi dalam literatur akademik Indonesia. Penelitian ini mengisi gap dengan memberikan blueprint arsitektur teknis yang menggabungkan NestJS sebagai backend framework dan LangChain.js untuk orkestrasi RAG pipeline. Integrasi ini mencakup pengelolaan service layers, dependency injection untuk LLM components, error handling, dan API design yang RESTful untuk endpoint chatbot.